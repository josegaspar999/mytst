\documentclass{llncs}
\usepackage{amsmath}
\usepackage{amsfonts}
\usepackage{amssymb}
\usepackage{graphicx}
%\usepackage{amsthm}
\graphicspath{{./figs/orig/}{./figs/}}


\newtheorem {mydef}{Definition}
\newtheorem {myprop}{Property}
\newtheorem {myobs}{Observation}
\newtheorem {myalg}{Algorithm}

 \setlength{\topmargin}{-10mm}
 \setlength{\textheight}{235mm}
 %\setlength{\topmargin}{0mm}
 \setlength{\headheight}{14pt}
 \setlength{\headsep}{10mm}
 \setlength{\textwidth}{150mm}
 \setlength{\footskip}{13mm}
 \setlength{\oddsidemargin}{5mm}
 \setlength{\evensidemargin}{5mm}

%\usepackage{float}
%\floatstyle{ruled}
%\newfloat{algorithm}{tbp}{loa}
%\floatname{algorithm}{Algorithm}


\title{Lens Classification for Discrete Cameras}
%
\titlerunning{Lens classification}  % abbreviated title (for running head)
%                                     also used for the TOC unless
%                                     \toctitle is used
%
\author{R. Galego\inst{1} \and R. Ferreira\inst{1} \and A. Bernardino\inst{1} \and E. Grossmann\inst{2} \and J. Gaspar\inst{1}
}
%
\authorrunning{R. Galego et al.} % abbreviated author list (for running head)
%
%%%% list of authors for the TOC (use if author list has to be modified)
\tocauthor{R. Galego, R. Ferreira, A. Bernardino, E. Grossmann and J. Gaspar}
%
\institute{Institute for Systems and Robotics, IST/UTL, Portugal\\
\email{\{rgalego,ricardo,alex,jag\}@isr.ist.utl.pt}, \\ 
%WWW home page:\texttt{http://www.isr.ist.utl.pt}
\and
Intel Corp., Menlo Park, USA\\
\email{etienne@egdn.net}, \\ 
}


% auxiliary commands
%
\newcommand{\R}[1] {\mathbb{R}^{#1}}
\newcommand{\Sphere}[1] {\mathbb{S}^{#1}}
\newcommand{\SO}[1] {\mathbb{SO}({#1})}

\newcommand{\Prob}[1] {\mathbb{P}\left[ #1 \right]}
\newcommand{\EV}[1] {\mathbb{E}\left[ #1 \right]}
\newcommand{\EVs}[1] {\mathbb{E}^2\left[ #1 \right]}

\newcommand{\p}[1] {\mathbf{#1}}
\def\pp {\p{p}}
\def\pq {\p{q}}
\def\acos {\operatorname{acos}}
\def\asin {\operatorname{asin}}
\def\Area {\operatorname{Area}}

\newcommand{\TODO}[1] {\textbf{*** #1  ***} }


%------------------------------------------------------------------------- 
% Document starts here
\begin{document}

\maketitle

%\begin{abstract}
%In this document we consider cameras having a cable of optic fibers forming a link between the image sensor (CCD) and a lens. We approach the problem of finding the lens type having estimated the sensor topology. We assume that we have a uniform density of fibers in the far-end of the cable.
%\end{abstract}

%------------------------------------------------------------------------- 

\begin{abstract}
In this document we consider cameras having a cable of optic fibers forming a link between the image sensor (CCD) and a lens. We approach the problem of finding the lens type having estimated the sensor topology. We assume that we have a uniform density of fibers in the far-end of the cable.
\end{abstract}


\section{Camera Model}

.... put here illustration (diagram) of the setup. Example CCD camera, cable of optic fibers, optic fibers frontal lens.

Consider that $r$ is the radial distance from the center of an imaging sensor, $l$ is the focal length and $\phi$ is the angle between the principal axis and the incoming ray.
In the following we assume  that we have three different lens: 

\begin{enumerate} 
	\item  perspective lens $\varphi = l \ atan(r)$
	\item equidistance projection lens, $\varphi = l \ r$
	\item orthogonal projection lens, $\varphi = l \ asin(r)$
\end{enumerate}

From now on the focal length $l$ will be not considered since it is a constant which cannot be observed. %Instead we will use f to denote pixel densities.


Suppose we have two kinds of pixels densities characterizing the cable of optic fibers (i) uniform in $uv$ (pixel coordinates), or (ii) constant for each circle, i.e. uniform in r. The uniform density in $uv$ implies a non uniform density in $r$, and vice versa (see figure\ref{fig:uniform_disp}).

\begin{figure}
\centering
\begin{tabular}{cc}
\includegraphics[width=6.0cm]{uniform_uv.eps} &
\includegraphics[width=6.0cm]{uniform_R.eps} 
\end{tabular}
\caption{Uniform density in uv, i.e. $f^{uv}(r,\theta)=$const (left). Uniform density considering just r, i.e. $f^r(r)=$const the marginal function obtained by accumulating for all values of $\theta$ (right).}
\label{fig:uniform_disp}
\end{figure}

%Let us start with the uniform density in uv. Noting that r=sqrt(u2+v2), the distribution corresponding to the uniform density in uv has a form of
%\begin{equation}
%f(r)= a*r*(u(r)-u(r-\max(r)))
%\end{equation}
%where $u(.)$ denotes the unit step / Heaviside function.

Given one probability density function (PDF), $f$, one obtains the cumulative distribution function (CDF), $F$, by integrating along the area wanted. Since we are using polar coordinates $(r,\alpha)$ we have to take into account the transformation $r$ correcting the integration elements $dr, d\alpha$:
% 
\begin{equation}
 %F(r, \alpha) = \int^{2\pi}_{0} \int^{R}_0 f(r, \alpha) r\ dr d\alpha
 F(r, \alpha) = \int^{\alpha}_{0} \int^{r}_0 f(r, \alpha) r\ dr d\alpha
\label{eq:CDF}
\end{equation}
%
where $\alpha \in [0,\ 2\pi]$ and $r \in [0,\ r_{max}]$.

The radial density, i.e. the PDF along the radial direction, is a single integral:
% 
\begin{equation}
 %F(r, \alpha) = \int^{2\pi}_{0} \int^{R}_0 f(r, \alpha) r\ dr d\alpha
 f_R(r) = \int^{2\pi}_{0} f(r, \alpha) d\alpha .
\label{eq:int_circle}
\end{equation}
%
Note that $F(r, 2\pi)= \int^r_{0} r f_R(r) dr$.


\section{Marginal PDF characterizing Pixel Densities}


In the case of uniform density in $uv$ one has $f^{uv}(r,\alpha)= C$.
Here the superscript represents the distribution.
Replacing $f$ by $f^{uv}$ in eq~\ref{eq:CDF} the CDF is
		$F_R(r)= C \pi r^2 $ .
Integrating over $\theta$ one obtains the marginal density in $r$ as
 \[ f_R(r)= dF(r) / dr . \]


Considering now the uniform density in $r$ one has $f^r(r, \alpha)= A / (2\pi r)$,
where $A$ denotes the number of points per radius $r$.
Replacing $f$ by $f^r$ in eq~\ref{eq:CDF} the CDF is $F_R(r)= A r$.
The marginal PDF is the derivative i.e.
%
\begin{equation}
f^r_R(r)= A.
\end{equation}


Note (a simpler way for derivations): Starting with the definition of marginalization
%
%\begin{equation}
$f_X(x)=\int_{-\inf}^{\inf} f(x,y)\ dy$
%\end{equation}
%
and considering polar coordinates, implies that one has to had $r$ in the integral, and finally obtain
%\begin{equation}
$f_R(r)=\int^{2\pi}_0 f(r,\alpha)r\ d\alpha$ .
%\end{equation}
%
As expected both ways lead to the same result.


The next figure~\ref{fig:histogram} shows the marginal distribution %of in both cases 
$F^{\{uv,R\}}_R(r)$ when we have a uniform density in $uv$ or when it is in $R$.

\begin{figure} \centering
\begin{tabular}{cc}
\includegraphics[width=6.0cm]{uniform_uv_hist.eps} &
\includegraphics[width=6.0cm]{uniform_R_hist.eps} 
\end{tabular}
\caption{Counting pixels in each circle (thin crown). Uniform distribution in uv (left) and uniform distribution in R (right).}
%On the left is the distribution of R when we have a uniform distribution in uv, on the right is the distribution of R when there is a uniform distribution in R.}
\label{fig:histogram}
\end{figure}


%\section{Lens Effect and Raxel Distributions}
%\section{Lens Effect and Raxel Densities}
\section{Lens Effect on the Density of Raxels given Pixels Distributed Uniformly}

The marginal distributions presented thus far characterize pixel distributions at the image plane. Now we want to characterize raxel distributions, i.e. the sensor topology including the geometric transformations created by the various types of lenses.

In other words, the lenses change the marginal distributions. From now on we will refer to marginal density as density.

The probability density function of a random variable X is given as $f_X(x)$. It is possible to calculate the probability density function of some variable $Y= g(X)$. This is also called a �change of variable� and is in practice used to generate a random variable of arbitrary shape $f_{g(X)} = f_Y$ using a known (for instance uniform) random number generator.

\begin{equation}
f_Y(y)=\left | \frac{d}{dy} (g^{-1}(y)) \right | f_x(g^{-1}(y))
\label{eq:wiki_copy}
\end{equation}

%\begin{tabular} {cc}
%In case of uniform distribution in R we have & The theoretical solution for the histograms with a uniform distribution in uv is, where \\
%1 & 2
%\end{tabular}

\vspace{5mm}

\begin{table*}[h] % place table ��here�� or at bottom of page
\centering
%\bigskip

\begin{tabular}{|l|p{1.5in}|p{2.0in}|}
\hline
& In case of uniform distribution in R we have:  & The theoretical solution for the histograms with a uniform distribution in uv is, where:  \\
\hline
 & $f_x=f^r_R(r)\propto 1$ &  $f_x=f^{uv}_R(r)\propto r$\\
\hline
$g(\Omega)=\tan(\Omega)$ & $f_y=\sec(\Omega)^2$ & $f_y=\sec(\Omega)^2\tan(\Omega)$ \\
$g(\Omega)=\Omega$ & $f_y=1$ & $f_y=\Omega$ \\
$g(\Omega)=\sin(\Omega)$ & $f_y=\cos(\Omega)$ & $f_y=\cos(\Omega)\sin(\Omega)$ \\
\hline
\end{tabular}
\end{table*}

For the simplest case of uniform distribution in R:

\begin{figure} \centering
\begin{tabular}{cc}
\includegraphics[width=6.0cm]{uniform_R_hist_perspective.eps} &
\includegraphics[width=6.0cm]{uniform_R_hist_ortogonal.eps} 
\end{tabular}
\caption{Histogram of orthogonal projection (right) and from perspective (left) at red are the theoretical values while in blue are the synthetic values. }
%\label{fig:histogram}
\end{figure}

Note the equidistant projection is not shown in here but the experimental vs theoretical results are also close.

In the next cases it is show a square, in the topology, instead of a circle, because it is easier to spot the differences between each case, however the histogram was made only using the biggest circle that could fit in the topology.
For the case we have a uniform distribution in uv ($f_X=f^{uv}_R(r)$):


\begin{figure} \centering
\begin{tabular}{cc}
\includegraphics[width=6.0cm]{uniform_uv_top_perspective.eps} &
\includegraphics[width=6.0cm]{uniform_uv_hist_perspective.eps} 
\end{tabular}
\caption{Angle distribution of a perspective lens, histogram made by collecting distances from all to the center of the mass (0,0). In red theoretical histogram, in blue real histogram. }
%\label{fig:histogram}
\end{figure}


\begin{figure} \centering
\begin{tabular}{cc}
\includegraphics[width=6.0cm]{uniform_uv_top_equidistant.eps} &
\includegraphics[width=6.0cm]{uniform_uv_hist_equidistant.eps} 
\end{tabular}
\caption{Angle distribution of an equidistance projection lens, histogram made by collecting distances from all to the center of the mass (0,0).}
%\label{fig:histogram}
\end{figure}

\begin{figure} \centering
\begin{tabular}{cc}
\includegraphics[width=6.0cm]{uniform_uv_top_ortogonal.eps} &
\includegraphics[width=6.0cm]{uniform_uv_hist_ortogonal.eps} 
\end{tabular}
\caption{Angle distribution of an orthogonal projection lens, histogram made by collecting distances from all to the center of the mass (0,0).  In red theoretical histogram, in blue real histogram.}
%\label{fig:histogram}
\end{figure}


%\section{Introduction}

%Traditional calibration assumes imaging sensors formed by pixels precisely placed in a rectangular grid, as the most common artificial vision systems are like that...
%
%BUT non-regular topologies proved to be interesting: fast computations with little resources

%Traditional imaging sensors are composed by pixels precisely placed to form rectangular grids, and thus look like calibrated sensors for many practical purposes such as localizing local extrema, edges or corners.

Traditional imaging sensors are formed by pixels precisely placed in a rectangular grid, and thus look like calibrated sensors for many practical purposes such as localizing local extrema, edges or corners.
In contrast, the most common imaging sensors found in nature are the compound eyes,
%found in many crustaceans and, in particular, in arachnids or insects,
% such as flies or bees
collections of individual photo cells which clearly do not form rectangular grids, but are very effective for solving %e.g. navigation tasks
%and certainly inspire the design of artificial systems \cite{Neumann04,Di09}.
various tasks at hand and thus have inspired the design of many artificial systems.
%
Volkel \etal studied several types of eyes and discussed the miniaturization of imaging systems~\cite{Volkel03}.
%
Neumann \etal ~\cite{Neumann04} proposed a compound eye vision sensor for 3D ego motion computation.
%
Recently, Micro-Electro-Mechanical Systems fabrication technologies were applied to build artificial compound eyes on planar surfaces~\cite{Di09}.

%Compound eyes are the type of eyes most commonly seen in many crustaceans and, in particular, in arachnids or insects, such as flies or bees
%Being composed by a set of individual photo cells, they allow obtaining wide fields of view and detecting fast movements with few resources. % computational resources.
%%However compound eyes have a small resolution to avoid an overflow of the animals neural system with too much data.
%%
%In recent years the scientific community started to focus on this type of vision aiming to mimic its advantages.
%
%V\"{o}lkel \etal studied several types of eyes and their repercussions into electronic imaging and micro-optical designs~\cite{Volkel03}.

In other words, novel fabrication technologies allow creating sensors with pixel arrangements (topologies) tuned for the tasks at hand.
%Many novel sensors will not have pixels forming regular rectangular grids.
%In some cases sensor fabrication may benefit of allowing the emergence of sensor topologies instead of rigidly imposing them.
%This precludes using traditional calibration methodologies~\cite{Agapito99,Zhang99,Sinha04}.
%In many cases this will preclude using traditional calibration methodologies~\cite{Agapito99,Zhang99,Sinha04}.
In the cases where the sensor topology is not a rectangular grid using traditional calibration methodologies~\cite{Agapito99,Zhang99,Sinha04} will not be possible.
%
Hence, the question arising here is: how to calibrate sensors with unknown topologies?
%
%Assuming that the sensors are mounted on mobile robots, the previous question can be reformulated as: can we calibrate an unknown topology sensor just with the data acquired by the sensor? 
%
In the case that the sensors are mounted on mobile robots the question can be restated as: can we calibrate an unknown topology of a moving sensor just with the data acquired by the sensor? 

Pierce and Kuipers %pioneered topological calibration by proposing methodologies and doing experiments showing 
have shown
that it is possible to reconstruct the topology of a group of sensors just by knowing their output~\cite{Kuipers97}.
%Pierce and Kuipers introduced the notion that is possible to reconstruct the topology of a group of sensors just by knowing their output~\cite{Kuipers97}.
They use natural %generic 
properties of an agent's world in order to infer the structure of its sensors.
%In particular they use two types of metrics: one based on the premise that usually adjacent sensors have similar values, a second metric is based on the frequency distributions of the sensors, since similar sensors should have the same frequency distributions.
%
Olsson \etal improved the methodologies introduced by Pierce and Kuipers by adding information distances and,
%in the case of de-scrambling a $20\times20$ pixels sensor,
%in particular, Hamming metrics~\cite{Olsson04}.
%More recently they compare the effects of different types of distance metrics~\cite{Olsson06}. 
in particular, Hamming metrics~\cite{Olsson04,Olsson06}.
They compute the position of several sensors of a Sony Aibo robot, which has, among other sensors, one camera sub-sampled to $8\times 8$ pixels.
%
%\TODO{Rewrite: Despite of the results they only use a $8\times8$ pixels camera, and they do not give information how does the information distance is related with the metric distance}.
%
%It was also stated by Hyvarinen \etal that a neurons (feature detectors) \emph{topography} can be obtained with a simple modification of the model of independent subspace analysis (ISA).
%
Hyvarinen \etal shown that imaging natural scenes allow defining a \emph{neuronal topography} %based on a simple modification of the model of 
using Independent Subspace Analysis~\cite{nis09}. % (ISA)~\cite{nis09}. %ISA is a generalization of a statistical generative model, the Independent Component Analysis, whose estimation boils down to sparse coding~\cite{nis09}.
%
Recently, Grossmann \etal~\cite{Grossmann10} proposed a method for calibrating a central imaging sensor based on a number of photocells. They need to know (estimate) a priori a function curve relating correlation (or information-distance) and distance-angles. % in order their algorithm works. 
Their algorithm has been tested on a small set of pixels (photocells), about one hundred, as otherwise the computation time and memory would be too large.

In this work we want to do auto-calibration of central sensors with a number of pixels orders of magnitude larger than \cite{Olsson06,Grossmann10}.
%\TODO{cite google uses the same tecnic for 17M points}
We approach the computational complexity with Multi Dimensional Scaling (MDS) like algorithms. A relatively old but very effective in the presence of noise free data is the Classical MDS~\cite{MDS}, based on Euclidean distances. Its goal is to find a representation of a data set on a given dimensionality from the knowledge of all interpoint distances.
Several new algorithms evolved from MDS, such as Isomap~\cite{Isomap}, where geodesic distances induced by a neighborhood graph are used instead of Euclidean distances. More recently Landmark Isomap~\cite{Landmark} was introduced, which uses only a subset of the all-to-all distances used in Isomap and has been proven to work on large scale datasets %(up to 18 million data points)~\cite{Google08}.
(millions of data points)~\cite{Google08} and constitute therefore a promising research direction. 
%approach to topological calibration.

%Another branch of the multi dimension scaling algorithms is the Self Organizing Maps (SOM), also known as Kohonen maps \cite{Kohonen81}. SOM are networks that use unsupervised training to preserve the topological properties of the input space. These maps need to have prior information of the topology, usually hexagonal or rectangular grid, to use in a neighborhood function. One characteristic of SOM is that it can be feed by different types of input metric and non metric.

The structure of the paper is the following:
in Sec.2 we study a simple black and white scenario and show there is a linear relationship between the correlation of the time series acquired by pairs of pixels and the inter-pixel angle;
in Sec.3 we describe Landmark-Isomap applied to topological calibration and propose
%a rotation and/or mirroring correction methodology;
a methodology for choosing a coordinate frame for the imaging sensor;
in Sec.4 we show some experimental results, and finally in Sec.5 we draw some conclusions.

%\input{sec1_related}
%\input{sec2_calibr}
%\input{sec3_corr}
%\section{Results}

%\subsection{Find sensor topology}
%Figure~\ref{fig:results1} shows ...
%\begin{figure}
\centering
\begin{tabular}{cccc}
\includegraphics[width=2.0cm]{moon_view_1.eps} & 
\includegraphics[width=2.0cm]{moon_view_4.eps} & 
\includegraphics[width=2.0cm]{moon_view_4.eps} &
\includegraphics[width=2.0cm]{moon_view_2.eps} \\
%\includegraphics[width=2.0cm]{empty.eps} & 
%\includegraphics[width=2.0cm]{empty.eps} & 
%\includegraphics[width=2.0cm]{empty.eps} &
%\includegraphics[width=2.0cm]{empty.eps} \\
(a) moon view &
(b) moon view &
(c) moon view &
(d) moon view \\
 frame 890 &
 frame 2703 &
 frame 2703 &
 frame 5215 %\\
\vspace{3mm} \\
\end{tabular}
%
%\begin{tabular}{cc}
%\includegraphics[width=3cm]{moon_unrot.eps} & 
%\includegraphics[width=3cm]{moon_rot.eps} \\
%(d) Topology result & (e) Topology arrange  %\\
%\vspace{3mm} \\
%\end{tabular}
%
\begin{tabular}{ccc}
\includegraphics[height=3.5cm]{topology_32_lines_unrot.eps} & 
\includegraphics[height=3cm]{topology_32_lines_rot.eps} &
\includegraphics[height=2cm]{topology_32_lines_crop_rot.eps} \\
(d) Topology result & (e) Rotated  & (f) Zoomed crop  %\\
\vspace{3mm} \\
\end{tabular}
%
\caption{
Topology using minimum bounding box (note: 14784 frames).
}
\label{fig:results1}
\end{figure}


%\subsection{Application of the topology finding}

%Nikon D5000, selected a central 100x100 pixels region (why? non-optimized Dijkstra / matlab?) Known topology, simple rectangular grid (topology errors clear to show locally and globally).
% ^^ & Calibrated camera (no need?)

In order to test the proposed topology estimation methodology two experiments have been conducted using a Nikon D5000 camera in video mode, selecting just a central region of 100x100 pixels, and thus having the ground truth of a sensor composed by square pixels forming a regular square grid.
%Data acquisition was performed at 24fps, about ten minutes for each experiment, while panning, tilting and rolling the camera.

%1) perfect conditions: video of the moon in a full moon night, approximately 8min video, 15k images (14784 images); pan tilt (and roll? roll does not influence as the full moon is very close to a circle...)
%
%2) non perfect conditions, hand held camera recording a video in the campus garden and car park; pan tilt and roll, and some translation; binarization helps when using short sequences (Grossmann10 and plot in sec2)


%--- Exp1: shuffling of pixels: no particular order of pixel-streams - in practice verified that despite the random selection of landmarks, the reconstructions are very similar \TODO{(Procrustes error less than ....)}

%\begin{figure}
%\centering
%\begin{tabular}{ccc}
%\includegraphics[width=3cm]{face_case_35.eps} & 
%\multicolumn{2}{c}{ \includegraphics[width=5cm]{face_shuffle.eps} }\\
%(a) Test image & \multicolumn{2}{c}{(b) $2 \times 4$ shuffling } %\\
%\vspace{3mm} \\
%\includegraphics[width=3cm]{face_case_32.eps} & 
%\includegraphics[width=3cm]{face_case_34.eps} & 
%\includegraphics[width=3cm]{face_case_30.eps} \\
%(c) $2 \times 4$ shuffling &
%(d) $10 \times 10$ shuffling &
%(e) $100 \times 100$ shuffling %\\
%\vspace{3mm} \\
%\includegraphics[width=3cm]{topology_file_res02_case_32.EPS} & \includegraphics[width=3cm]{topology_file_res02_case_34.EPS} & \includegraphics[width=3cm]{topology_file_res02_case_30.EPS} \\
%(f) Reconstruction after & (g) Reconstruction after & (h) Reconstruction after \\
% $2 \times 4$ shuffling &
% $10 \times 10$ shuffling &
% $100 \times 100$ shuffling %\\
%\vspace{3mm} \\
%\end{tabular}
%\caption{
%Topology estimation applied to the reconstruction of a shuffled $100 \times 100$ sensor.
%%Topology estimation applied to reordering a shuffled sensor.
%%Topology estimation applied to the reconstruction of an image.
%Test image before shuffling (a).
%The pixels of the sensor are shuffled in (i) $2 \times 4$ blocks, (ii) $10 \times 10$ blocks, or (iii) $100 \times 100$ (all) pixels, as illustrated on the test image, (b,c), (d) or (e), respectively.
%%The pixels of a test image (a) are shuffled in $2 \times 4$ blocks (b,c), $10\times\10$ blocks, or $100\times\100$ (all) pixels (d). 
%Each of the three shufflings is then subject to topology estimation, and the resulting topology applied to reconstruct the test image (f,g,h).
%%The estimated sensor topology, applied to each of the three cases, in then used to reconstruct the test image (e,f,g).
%}
%\label{fig:results2}
%\end{figure}


\begin{figure}[t]
\centering
\begin{tabular}{ccc}
\includegraphics[height=3.0cm]{moon_view_6frames.eps} &
%\includegraphics[height=3.2cm]{topology_32_lines_unrot.eps} & 
\includegraphics[height=3.2cm]{topology_32_lines_unrot2.eps} & 
\includegraphics[height=2.7cm]{topology_32_lines_rot2.eps} \\
%\parbox{3cm}{\centering \small (a) Calibration frames 890, 2712, 4281 and 5263} &
\parbox{3cm}{\centering \small (a) 6 of 14784 calibration frames} &
\parbox{3cm}{\centering \small (b) Topology found} & 
\parbox{5cm}{\centering \small (c) Topology found, rotated (Eq.\ref{eq:rot_mirror}) and top-right zoomed} %\\ 
\vspace{3mm} \\
\end{tabular}
%
\begin{tabular}{ccccc}
\includegraphics[width=2.0cm]{car_case_35.eps} & 
\includegraphics[width=2.0cm]{face_shuffle2.eps} &
\includegraphics[width=2.0cm]{car_case_32.eps} & 
\includegraphics[width=2.0cm]{car_case_34.eps} & 
\includegraphics[width=2.0cm]{car_case_30.eps} \\
\parbox{2cm}{\centering \small (d) Test image} &
\parbox{2cm}{\centering \small (e) $2 \times 4$ permutation} &
\parbox{2cm}{\centering \small (f) $2 \times 4$ permutation} &
\parbox{2cm}{\centering \small (g) $10 \times 10$ permutation} &
\parbox{2cm}{\centering \small (h) $100 \times 100$ permutation} %\\
\vspace{3mm} \\
\end{tabular}
%
\begin{tabular}{ccc}
%\includegraphics[width=2.0cm]{topology_file_res02_case_32_v1.eps} & 
%\includegraphics[width=2.0cm]{topology_file_res02_case_34_v1.eps} & 
%\includegraphics[width=2.0cm]{topology_file_res02_case_30_v1.eps} \\
\includegraphics[width=2.0cm]{topology_find_res02_case_32_rot_mirror.eps} & 
\includegraphics[width=2.0cm]{topology_find_res02_case_34_rot_mirror.eps} & 
\includegraphics[width=2.0cm]{topology_find_res02_case_30_rot_mirror.eps} \\
\parbox{3.0cm}{\centering \small (i) Reconstruction after $2 \times 4$ permutation} &
\parbox{3.0cm}{\centering \small (j) Reconstruction after $10 \times 10$ permutation} &
\parbox{3.2cm}{\centering \small (k) Reconstruction after $100 \times 100$ permutation} %\\
\vspace{3mm} \\
\end{tabular}
\caption{
%Topology using minimum bounding box (note: 14784 frames).
%Topology estimation applied to the reconstruction of a shuffled $100 \times 100$ sensor.
%Topology estimation applied to reordering a shuffled sensor.
%Topology estimation applied to the reconstruction of an image.
Moon images used to estimate the topology of a $100 \times 100$ sensor (a).
Estimated topology after random permutation of the pixel-streams (b,c).
%Test image before shuffling (d).
%The pixels of the sensor are shuffled in (i) $2 \times 4$ blocks, (ii) $10 \times 10$ blocks, or (iii) $100 \times 100$ (all) pixels, as illustrated on the test image, (e,f), (g) or (h), respectively.
Test image before permutation (d).
%Permutations of pixel-streams in $2 \times 4$ blocks (e), $10 \times 10$ blocks, or $100 \times 100$ (all) pixels, illustrated on the test image, (f), (g) or (h), respectively.
%Each of the three permutations is then subject to topology estimation, and the resulting topology applied to reconstruct the test image (i,j,k).
Permutations of pixel-streams in $2 \times 4$ blocks, $10 \times 10$ blocks, or $100 \times 100$ (all) pixels, illustrated on the test image, (e,f,g,h).
Estimated topology applied to reconstruct the test image after the three permutations (i,j,k).
}
\label{fig:results2}
\end{figure}


In the first experiment the camera was pointed to the moon, in a full-moon night, to obtain calibration data (see Fig.~\ref{fig:results2}(a)). The data acquisition was performed at 24fps for about ten minutes (14784 frames), while panning and tilting the camera.
%
Figures~\ref{fig:results2}(b,c) show the estimated topology, approximately forming a regular square grid, close to the ground truth.
%
%Manifold learning toolbox modified to receive distances instead of the pixel streams (so we can use a correlation based distance instead of a simple euclidean distance)
% ^^ time delay inter-pixel is the most informative data?
%
A test image, acquired in daylight (Fig.~\ref{fig:results2}(d)), was then used to illustrate more clearly that the sequencing of the pixel-streams (see pixel permutations in Figs.~\ref{fig:results2}(e,f,g,h)) does not influence the perceptual quality of the estimated topology and image reconstruction (Figs.~\ref{fig:results2}(i,j,k)).
%
Despite having obtained the results in Figs.~\ref{fig:results2}(i,j,k) with different calibrations, and thus subject to different random selections of landmark pixels, the differences of the estimated topologies are small.
Using a 2D Procrustes, to register the three reconstructed topologies with a square 1-pixel-steps grid, resulted in 
%average relative pixel localization errors
%~\footnote{$e= \sum\nolimits_i {\sqrt {xi' - xi} /N/W}$, where $W$ is the image width} of $3.52\%$, $3.35\%$ and $3.92\%$, respectively. 
%of $11.14\%$, $11.08\%$ and $11.00\%$, and
inter-pixel (four nearest neighbors) distance-error distribution with a standard deviation of $0.566$, $0.563$ and $0.563$ pixels, respectively. %, around the optimal location.

%--- Exp2: calibration given a sequence of the garden
% two critical aspects: (i) uniform edge directions or uniform motion (vs longer sequences?), (ii) binarization to make sharper (lesser ambiguous) pixel-stream transitions
%.... uniform edge directions distribution similar to what was found in the moon sequence, hence the selection of a garden sequence
%From the bibliography it is known that 
%
%results: (i) gray level imply scaling of diagonal directions
%(ii) lost aspect ratio, maybe due to bias short sequence, not perfectly uniform distribution of pan, tilt and roll

\begin{figure}
\centering
%
\begin{tabular}{cccc}
\includegraphics[width=2.0cm]{garden_0123_crop_000.eps} & 
\includegraphics[width=2.0cm]{garden_0123_crop_025.eps} & 
\includegraphics[width=2.0cm]{garden_0123_crop_6105.eps} & 
\includegraphics[width=2.0cm]{garden_0123_crop_13209.eps} \\
\parbox{2cm}{\centering \small (a) Calibration frame 1} &
\parbox{2cm}{\centering \small (b) Calibration frame 25} &
\parbox{2cm}{\centering \small (c) Calibration frame 6105} &
\parbox{2cm}{\centering \small (d) Calibration frame 13209} \\
\end{tabular}
%
\begin{tabular}{cccc}
\vspace{3mm} \\
\includegraphics[width=2cm]{garden_0123_crop_6105_bw.eps} &
\includegraphics[width=2cm]{garden_0123_crop_13209_bw.eps} & \includegraphics[width=2cm]{topology_find_res02_case_31_150_lines.eps} & \includegraphics[width=2cm]{topology_find_res02_case_31_150_rot_mirror.eps} \\
%
\parbox{2cm}{\centering \small (e) Frame 6105 binarized} &
\parbox{2cm}{\centering \small (f) Frame 13209 binarized} &
\parbox{2cm}{\centering \small (g) Topology found} &
\parbox{2cm}{\centering \small (h) Frame 13209 reconstructed} \\
%
\vspace{3mm} \\
\end{tabular}
%
\caption{
Topology estimation applied to 17448 images random images of the University campus
}
\label{fig:results3}
\end{figure}


In the second experiment, the main purpose is to explore more general calibration scenarios, while keeping the topology estimation accurate.
%
In the moon sequence, edges are clearly defined and have directions distributed uniformly despite of the discretization due to the non-infinitesimal pixel size.
%
Hence we considered the garden %and car park 
scenario, Fig.~\ref{fig:results3}(a,b,c,d), where the vegetation also provides many edge directions. For this data set we filmed about twelve minutes, at 24fps, having acquired 17448 frames. In this case we do pan and tilt motions, as well as roll and translation. % although most of the edges are vertical and horizontal. %, 
%which can be further extended by rolling the camera, 
%and image binarization augments edges sharpness.
%
%After the binarization the 
Figure~\ref{fig:results3}(g) shows the topology reconstruction considering binary level pixel-streams. Some images used by the algorithm can be seen in Figs.~\ref{fig:results3}(e,f).
%
Despite the complexity of scene texture
%features in the environment observed 
we do not use a look up table as required in \cite{Grossmann10}. %One observes that the image corners are stretched out, which is due to an overestimation of inter-pixel distances in the vertical direction. This can be due to a bias in the dataset which may contain more horizontal edges than vertical.
%
%Fig g is rotated and mirrored
%
%Fig h is corrected 
%
%The rotation and mirror in 
%
%Observing Figs.~\ref{fig:results3}(g,h) simultaneously, one can detect that the topology suffered a mirror effect, this is explained because 
%
Note that Fig.~\ref{fig:results3}(g) is the direct result of the Landmark Isomap, while 
Fig.~\ref{fig:results3}(h) is the result after using Eq.\ref{eq:rot_mirror}, and thus show a detected and corrected mirror effect.
%
%which detects if the output of the multi scaling algorithm is affected by a mirror effect.
%
%By binarizing the images, Figs.~\ref{fig:results3}(f,g), one augments image edges distinctiveness and improves the accuracy of reconstruction, Fig.~\ref{fig:results3}(h). This beneficial effect of image binarization is in accordance with related work \cite{Grossmann10}, where was shown experimentally that reducing the brightness levels of images would allow using shorter calibration sequences.


------

Proposed a simple auto-calibration methodology for an arbitrary central projection camera, in this work an optic fibers bundle. Proved that the relationship between correlation and inter-pixel angle is approximately linear for a camera rotating in the center of a dark spherical-surface with a light circular (hub-cap) source

-------

%\section{Conclusions and Future work}

%\section*{Acknowledgments}
%
%%\noindent To be filled.
%\noindent This work has been partially supported
%by the FCT project PEst-OE / EEI / LA0009 / 2011,
%by the FCT project PTDC / EEACRO / 105413 / 2008  DCCAL,
%and
%by the project High Definition Analytics (HDA), QREN - I\&D em Co-Promo\c{c}\~{a}o 13750.
%%
%%We thank Jo\~{a}o Mendanha Dias, IST Lisbon, the help to build the microscopic input, and Ricardo Nunes, IST/ISR Lisbon, the help to build the setup based on a cable of optic fibers.

\bibliographystyle{splncs03}
%\bibliography{secz}
\end{document}
