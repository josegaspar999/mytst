\begin{figure*}[!ht]
\centering
\begin{tabular}{cccc}
%\includegraphics[width=1.4\columnwidth]{setup_v1.eps} 
\includegraphics[height=4.5cm]{dcam_diagr.eps} &
\hspace{0.5cm}
\includegraphics[height=4.5cm]{dcam_coord_frames.eps} &
\hspace{1cm}
%\includegraphics[height=4.5cm]{dcam_imgs.eps} &
\includegraphics[height=4.5cm]{dcam_imgs_grey.eps} &
%\includegraphics[width=.4\columnwidth]{raxel_v2.eps} \\
%\includegraphics[height=4.5cm]{raxel_v2.eps} \\
\hspace{1cm}
\includegraphics[height=4.5cm]{dcam_raxels.eps} \\
(a) Camera and basis & (b) Coordinate frames & (c) Images at D and E & (d) Projection model \\
\end{tabular}
%\caption{An overview of the setup, and some examples of the images acquired in each step}
\caption{Model of a discrete camera mounted on a pan-tilt basis. The optic-fiber bundle, points E to D in (a), twists the input image (d). Vectors $v_1$ and $v_2$ allow computing a unity transform to obtain $\{4\}$ (b). Projection model and raxels notation (d).}
%\caption{Characterizing raxels.}
\label{fig:setup}
\end{figure*}
