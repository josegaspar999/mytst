\section{Marginal PDF characterizing Pixel Densities}

Suppose we have two kinds of pixels densities characterizing the cable optic fibers (i) uniform in $uv$ (pixel coordinates), or constant marginal density for each circle, i.e. uniform in r. The uniform density in $uv$ implies a non uniform density in $r$, and vice versa (see figure\ref{fig:uniform_disp}).

\begin{figure}
\centering
\begin{tabular}{cc}
\includegraphics[width=6.0cm]{uniform_uv.eps} &
\includegraphics[width=6.0cm]{uniform_R.eps} \\
(a) & (b) \\
\includegraphics[height=4.0cm]{uniform_uv_hist.eps} &
\includegraphics[height=4.0cm]{uniform_R_hist.eps} \\
(c) & (d) 
\end{tabular}
\caption{
Uniform density in uv, i.e. $f_{uv}(x,y)=$const (a).
Uniform density considering just r, i.e. $f_r(r)=$const the marginal function obtained by accumulating for all values of $\alpha$ (b).
Histogram of (a) and (b) along the radius, considering a thin crowd for each $r$, shown in (c) and (d), respectively.
}
\label{fig:uniform_disp}
\end{figure}

%Let us start with the uniform density in uv. Noting that r=sqrt(u2+v2), the distribution corresponding to the uniform density in uv has a form of
%\begin{equation}
%f(r)= a*r*(u(r)-u(r-\max(r)))
%\end{equation}
%where $u(.)$ denotes the unit step / Heaviside function.


The probability density function of a random variable X is given as $f_X(x)$. It is possible to calculate the probability density function of some variable $Y= g(X)$. This is also called a change of variable and is in practice used to generate a random variable of arbitrary shape $f_{g(X)} = f_Y$ using a known (for instance uniform) random number generator.

\begin{equation}
f_Y(y)=\left | \frac{d}{dy} (g^{-1}(y)) \right | f_x(g^{-1}(y))
\label{eq:wiki_copy}
\end{equation}

%\begin{tabular} {cc}
%In case of uniform distribution in R we have & The theoretical solution for the histograms with a uniform distribution in uv is, where \\
%1 & 2
%\end{tabular}

Looking at the case we assume uniform density in uv, we have to convert the density in to the polar coordinates, the transformation from Cartesian coordinates into polar is:

\begin{equation} 
\begin{array}{c}
x=R\cos(\alpha) \\
y=R\sin(\alpha)
\end{array}
\label{eq:xy2polar}
\end{equation}
%
considering $J$ the Jacobian matrix of equation~\ref{eq:xy2polar} we have:

\begin{equation}
|J(r,\alpha)|= \left |
\begin{array} {cc}
\cos(\alpha) & \sin(\alpha) \\
-r\sin(\alpha) & r\cos(\alpha)
\end{array} \right |=r
\end{equation}

As we assume that we have a uniform density in uv, we can consider that it is one, $f_{ux}(x,y)=1$, replacing the the density and the Jacobian in equation \ref{eq:wiki_copy} the result is:

\begin{equation}
f_{r\alpha}(r,\alpha)=|J(r,\alpha)|f_{ux}(r\cos(\alpha), r\sin(\alpha))=r
\end{equation} 


%The distribution made by a density function is calculated using an integral in the area wanted, however since we are using the polar coordinates we have to take into account the transformation $r$.
%%
%
%\begin{equation}
 %F(r, \alpha) = \int^{2\pi}_{0} \int^{R}_0 f(r, \alpha) r\ dr d\alpha
%\label{eq:int_circle}
%\end{equation}
%%
%$f$ corresponds to the density function, in the case of uniform distribution in $uv$ the $f^{uv}(r,\alpha)=c$, being $c$ the value of the uniform distribution and the superscript represents the distribution, replacing in eq~\ref{eq:int_circle} the distribution will be $c\pi R^2$.
%%
%For the uniform distribution in $R$ the density function corresponds to $f^R(r, \alpha)=\frac{a}{2\pi r}$, being $a$ the number of points per radius $r$. Replacing in  eq~\ref{eq:int_circle} the distribution will be $a R$.
%However since we want only the marginal distribution in $r$, $f_R(r)=\frac{F(r,\alpha)}{d_r}$. In the case of the uniform distribution in $uv$ the marginal distribution will be 
%%
%\begin{equation}
%f^{uv}_R(r)=c2\pi r
%\end{equation}
%%
%in the case of uniform distribution in R it will be a constant 
%%
%\begin{equation}
%f^{R}_R(r)=a.
%\end{equation}

Since the definition of marginalization is:
%
\begin{equation}
f_X(x)=\int_{-\inf}^{\inf} f(x,y)\ dy
\end{equation}
%
one can compute the marginal density in $R$.
%
\begin{equation}
f_R(r)=\int^{2\pi}_0 f(r,\alpha) d\alpha =2\pi r
\end{equation}
%
Note that for the second case we already know $f_R(r)$ since we consider that is has an uniform marginal density in $R$.

Figures~\ref{fig:uniform_disp}(c) and (d) show the marginal density of in both cases $f^{\{uv,R\}}_R(r)$, when we have a uniform in uv or when it is in R.

%\begin{figure}
%\centering
%\begin{tabular}{cc}
%\includegraphics[width=6.0cm]{uniform_uv_hist.eps} &
%\includegraphics[width=6.0cm]{uniform_R_hist.eps} 
%\end{tabular}
%\caption{On the left is the distribution of R when we have a uniform distribution in uv, on the right is the distribution of R when there is a uniform distribution in R.}
%\label{fig:histogram}
%\end{figure}
