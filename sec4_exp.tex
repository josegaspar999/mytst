\section{Results}

%\subsection{Find sensor topology}
%Figure~\ref{fig:results1} shows ...
%\begin{figure}
\centering
\begin{tabular}{cccc}
\includegraphics[width=2.0cm]{moon_view_1.eps} & 
\includegraphics[width=2.0cm]{moon_view_4.eps} & 
\includegraphics[width=2.0cm]{moon_view_4.eps} &
\includegraphics[width=2.0cm]{moon_view_2.eps} \\
%\includegraphics[width=2.0cm]{empty.eps} & 
%\includegraphics[width=2.0cm]{empty.eps} & 
%\includegraphics[width=2.0cm]{empty.eps} &
%\includegraphics[width=2.0cm]{empty.eps} \\
(a) moon view &
(b) moon view &
(c) moon view &
(d) moon view \\
 frame 890 &
 frame 2703 &
 frame 2703 &
 frame 5215 %\\
\vspace{3mm} \\
\end{tabular}
%
%\begin{tabular}{cc}
%\includegraphics[width=3cm]{moon_unrot.eps} & 
%\includegraphics[width=3cm]{moon_rot.eps} \\
%(d) Topology result & (e) Topology arrange  %\\
%\vspace{3mm} \\
%\end{tabular}
%
\begin{tabular}{ccc}
\includegraphics[height=3.5cm]{topology_32_lines_unrot.eps} & 
\includegraphics[height=3cm]{topology_32_lines_rot.eps} &
\includegraphics[height=2cm]{topology_32_lines_crop_rot.eps} \\
(d) Topology result & (e) Rotated  & (f) Zoomed crop  %\\
\vspace{3mm} \\
\end{tabular}
%
\caption{
Topology using minimum bounding box (note: 14784 frames).
}
\label{fig:results1}
\end{figure}


%\subsection{Application of the topology finding}

%Nikon D5000, selected a central 100x100 pixels region (why? non-optimized Dijkstra / matlab?) Known topology, simple rectangular grid (topology errors clear to show locally and globally).
% ^^ & Calibrated camera (no need?)

In order to test the proposed topology estimation methodology two experiments have been conducted using a Nikon D5000 camera in video mode, selecting just a central region of 100x100 pixels, and thus having the ground truth of a sensor composed by square pixels forming a regular square grid.
%Data acquisition was performed at 24fps, about ten minutes for each experiment, while panning, tilting and rolling the camera.

%1) perfect conditions: video of the moon in a full moon night, approximately 8min video, 15k images (14784 images); pan tilt (and roll? roll does not influence as the full moon is very close to a circle...)
%
%2) non perfect conditions, hand held camera recording a video in the campus garden and car park; pan tilt and roll, and some translation; binarization helps when using short sequences (Grossmann10 and plot in sec2)


%--- Exp1: shuffling of pixels: no particular order of pixel-streams - in practice verified that despite the random selection of landmarks, the reconstructions are very similar \TODO{(Procrustes error less than ....)}

%\begin{figure}
%\centering
%\begin{tabular}{ccc}
%\includegraphics[width=3cm]{face_case_35.eps} & 
%\multicolumn{2}{c}{ \includegraphics[width=5cm]{face_shuffle.eps} }\\
%(a) Test image & \multicolumn{2}{c}{(b) $2 \times 4$ shuffling } %\\
%\vspace{3mm} \\
%\includegraphics[width=3cm]{face_case_32.eps} & 
%\includegraphics[width=3cm]{face_case_34.eps} & 
%\includegraphics[width=3cm]{face_case_30.eps} \\
%(c) $2 \times 4$ shuffling &
%(d) $10 \times 10$ shuffling &
%(e) $100 \times 100$ shuffling %\\
%\vspace{3mm} \\
%\includegraphics[width=3cm]{topology_file_res02_case_32.EPS} & \includegraphics[width=3cm]{topology_file_res02_case_34.EPS} & \includegraphics[width=3cm]{topology_file_res02_case_30.EPS} \\
%(f) Reconstruction after & (g) Reconstruction after & (h) Reconstruction after \\
% $2 \times 4$ shuffling &
% $10 \times 10$ shuffling &
% $100 \times 100$ shuffling %\\
%\vspace{3mm} \\
%\end{tabular}
%\caption{
%Topology estimation applied to the reconstruction of a shuffled $100 \times 100$ sensor.
%%Topology estimation applied to reordering a shuffled sensor.
%%Topology estimation applied to the reconstruction of an image.
%Test image before shuffling (a).
%The pixels of the sensor are shuffled in (i) $2 \times 4$ blocks, (ii) $10 \times 10$ blocks, or (iii) $100 \times 100$ (all) pixels, as illustrated on the test image, (b,c), (d) or (e), respectively.
%%The pixels of a test image (a) are shuffled in $2 \times 4$ blocks (b,c), $10\times\10$ blocks, or $100\times\100$ (all) pixels (d). 
%Each of the three shufflings is then subject to topology estimation, and the resulting topology applied to reconstruct the test image (f,g,h).
%%The estimated sensor topology, applied to each of the three cases, in then used to reconstruct the test image (e,f,g).
%}
%\label{fig:results2}
%\end{figure}


\begin{figure}[t]
\centering
\begin{tabular}{ccc}
\includegraphics[height=3.0cm]{moon_view_6frames.eps} &
%\includegraphics[height=3.2cm]{topology_32_lines_unrot.eps} & 
\includegraphics[height=3.2cm]{topology_32_lines_unrot2.eps} & 
\includegraphics[height=2.7cm]{topology_32_lines_rot2.eps} \\
%\parbox{3cm}{\centering \small (a) Calibration frames 890, 2712, 4281 and 5263} &
\parbox{3cm}{\centering \small (a) 6 of 14784 calibration frames} &
\parbox{3cm}{\centering \small (b) Topology found} & 
\parbox{5cm}{\centering \small (c) Topology found, rotated (Eq.\ref{eq:rot_mirror}) and top-right zoomed} %\\ 
\vspace{3mm} \\
\end{tabular}
%
\begin{tabular}{ccccc}
\includegraphics[width=2.0cm]{car_case_35.eps} & 
\includegraphics[width=2.0cm]{face_shuffle2.eps} &
\includegraphics[width=2.0cm]{car_case_32.eps} & 
\includegraphics[width=2.0cm]{car_case_34.eps} & 
\includegraphics[width=2.0cm]{car_case_30.eps} \\
\parbox{2cm}{\centering \small (d) Test image} &
\parbox{2cm}{\centering \small (e) $2 \times 4$ permutation} &
\parbox{2cm}{\centering \small (f) $2 \times 4$ permutation} &
\parbox{2cm}{\centering \small (g) $10 \times 10$ permutation} &
\parbox{2cm}{\centering \small (h) $100 \times 100$ permutation} %\\
\vspace{3mm} \\
\end{tabular}
%
\begin{tabular}{ccc}
%\includegraphics[width=2.0cm]{topology_file_res02_case_32_v1.eps} & 
%\includegraphics[width=2.0cm]{topology_file_res02_case_34_v1.eps} & 
%\includegraphics[width=2.0cm]{topology_file_res02_case_30_v1.eps} \\
\includegraphics[width=2.0cm]{topology_find_res02_case_32_rot_mirror.eps} & 
\includegraphics[width=2.0cm]{topology_find_res02_case_34_rot_mirror.eps} & 
\includegraphics[width=2.0cm]{topology_find_res02_case_30_rot_mirror.eps} \\
\parbox{3.0cm}{\centering \small (i) Reconstruction after $2 \times 4$ permutation} &
\parbox{3.0cm}{\centering \small (j) Reconstruction after $10 \times 10$ permutation} &
\parbox{3.2cm}{\centering \small (k) Reconstruction after $100 \times 100$ permutation} %\\
\vspace{3mm} \\
\end{tabular}
\caption{
%Topology using minimum bounding box (note: 14784 frames).
%Topology estimation applied to the reconstruction of a shuffled $100 \times 100$ sensor.
%Topology estimation applied to reordering a shuffled sensor.
%Topology estimation applied to the reconstruction of an image.
Moon images used to estimate the topology of a $100 \times 100$ sensor (a).
Estimated topology after random permutation of the pixel-streams (b,c).
%Test image before shuffling (d).
%The pixels of the sensor are shuffled in (i) $2 \times 4$ blocks, (ii) $10 \times 10$ blocks, or (iii) $100 \times 100$ (all) pixels, as illustrated on the test image, (e,f), (g) or (h), respectively.
Test image before permutation (d).
%Permutations of pixel-streams in $2 \times 4$ blocks (e), $10 \times 10$ blocks, or $100 \times 100$ (all) pixels, illustrated on the test image, (f), (g) or (h), respectively.
%Each of the three permutations is then subject to topology estimation, and the resulting topology applied to reconstruct the test image (i,j,k).
Permutations of pixel-streams in $2 \times 4$ blocks, $10 \times 10$ blocks, or $100 \times 100$ (all) pixels, illustrated on the test image, (e,f,g,h).
Estimated topology applied to reconstruct the test image after the three permutations (i,j,k).
}
\label{fig:results2}
\end{figure}


In the first experiment the camera was pointed to the moon, in a full-moon night, to obtain calibration data (see Fig.~\ref{fig:results2}(a)). The data acquisition was performed at 24fps for about ten minutes (14784 frames), while panning and tilting the camera.
%
Figures~\ref{fig:results2}(b,c) show the estimated topology, approximately forming a regular square grid, close to the ground truth.
%
%Manifold learning toolbox modified to receive distances instead of the pixel streams (so we can use a correlation based distance instead of a simple euclidean distance)
% ^^ time delay inter-pixel is the most informative data?
%
A test image, acquired in daylight (Fig.~\ref{fig:results2}(d)), was then used to illustrate more clearly that the sequencing of the pixel-streams (see pixel permutations in Figs.~\ref{fig:results2}(e,f,g,h)) does not influence the perceptual quality of the estimated topology and image reconstruction (Figs.~\ref{fig:results2}(i,j,k)).
%
Despite having obtained the results in Figs.~\ref{fig:results2}(i,j,k) with different calibrations, and thus subject to different random selections of landmark pixels, the differences of the estimated topologies are small.
Using a 2D Procrustes, to register the three reconstructed topologies with a square 1-pixel-steps grid, resulted in 
%average relative pixel localization errors
%~\footnote{$e= \sum\nolimits_i {\sqrt {xi' - xi} /N/W}$, where $W$ is the image width} of $3.52\%$, $3.35\%$ and $3.92\%$, respectively. 
%of $11.14\%$, $11.08\%$ and $11.00\%$, and
inter-pixel (four nearest neighbors) distance-error distribution with a standard deviation of $0.566$, $0.563$ and $0.563$ pixels, respectively. %, around the optimal location.

%--- Exp2: calibration given a sequence of the garden
% two critical aspects: (i) uniform edge directions or uniform motion (vs longer sequences?), (ii) binarization to make sharper (lesser ambiguous) pixel-stream transitions
%.... uniform edge directions distribution similar to what was found in the moon sequence, hence the selection of a garden sequence
%From the bibliography it is known that 
%
%results: (i) gray level imply scaling of diagonal directions
%(ii) lost aspect ratio, maybe due to bias short sequence, not perfectly uniform distribution of pan, tilt and roll

\begin{figure}
\centering
%
\begin{tabular}{cccc}
\includegraphics[width=2.0cm]{garden_0123_crop_000.eps} & 
\includegraphics[width=2.0cm]{garden_0123_crop_025.eps} & 
\includegraphics[width=2.0cm]{garden_0123_crop_6105.eps} & 
\includegraphics[width=2.0cm]{garden_0123_crop_13209.eps} \\
\parbox{2cm}{\centering \small (a) Calibration frame 1} &
\parbox{2cm}{\centering \small (b) Calibration frame 25} &
\parbox{2cm}{\centering \small (c) Calibration frame 6105} &
\parbox{2cm}{\centering \small (d) Calibration frame 13209} \\
\end{tabular}
%
\begin{tabular}{cccc}
\vspace{3mm} \\
\includegraphics[width=2cm]{garden_0123_crop_6105_bw.eps} &
\includegraphics[width=2cm]{garden_0123_crop_13209_bw.eps} & \includegraphics[width=2cm]{topology_find_res02_case_31_150_lines.eps} & \includegraphics[width=2cm]{topology_find_res02_case_31_150_rot_mirror.eps} \\
%
\parbox{2cm}{\centering \small (e) Frame 6105 binarized} &
\parbox{2cm}{\centering \small (f) Frame 13209 binarized} &
\parbox{2cm}{\centering \small (g) Topology found} &
\parbox{2cm}{\centering \small (h) Frame 13209 reconstructed} \\
%
\vspace{3mm} \\
\end{tabular}
%
\caption{
Topology estimation applied to 17448 images random images of the University campus
}
\label{fig:results3}
\end{figure}


In the second experiment, the main purpose is to explore more general calibration scenarios, while keeping the topology estimation accurate.
%
In the moon sequence, edges are clearly defined and have directions distributed uniformly despite of the discretization due to the non-infinitesimal pixel size.
%
Hence we considered the garden %and car park 
scenario, Fig.~\ref{fig:results3}(a,b,c,d), where the vegetation also provides many edge directions. For this data set we filmed about twelve minutes, at 24fps, having acquired 17448 frames. In this case we do pan and tilt motions, as well as roll and translation. % although most of the edges are vertical and horizontal. %, 
%which can be further extended by rolling the camera, 
%and image binarization augments edges sharpness.
%
%After the binarization the 
Figure~\ref{fig:results3}(g) shows the topology reconstruction considering binary level pixel-streams. Some images used by the algorithm can be seen in Figs.~\ref{fig:results3}(e,f).
%
Despite the complexity of scene texture
%features in the environment observed 
we do not use a look up table as required in \cite{Grossmann10}. %One observes that the image corners are stretched out, which is due to an overestimation of inter-pixel distances in the vertical direction. This can be due to a bias in the dataset which may contain more horizontal edges than vertical.
%
%Fig g is rotated and mirrored
%
%Fig h is corrected 
%
%The rotation and mirror in 
%
%Observing Figs.~\ref{fig:results3}(g,h) simultaneously, one can detect that the topology suffered a mirror effect, this is explained because 
%
Note that Fig.~\ref{fig:results3}(g) is the direct result of the Landmark Isomap, while 
Fig.~\ref{fig:results3}(h) is the result after using Eq.\ref{eq:rot_mirror}, and thus show a detected and corrected mirror effect.
%
%which detects if the output of the multi scaling algorithm is affected by a mirror effect.
%
%By binarizing the images, Figs.~\ref{fig:results3}(f,g), one augments image edges distinctiveness and improves the accuracy of reconstruction, Fig.~\ref{fig:results3}(h). This beneficial effect of image binarization is in accordance with related work \cite{Grossmann10}, where was shown experimentally that reducing the brightness levels of images would allow using shorter calibration sequences.


------

Proposed a simple auto-calibration methodology for an arbitrary central projection camera, in this work an optic fibers bundle. Proved that the relationship between correlation and inter-pixel angle is approximately linear for a camera rotating in the center of a dark spherical-surface with a light circular (hub-cap) source

-------
