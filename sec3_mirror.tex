\section{Mirror effect}

As referred, MDS (and derived methods) provide a reconstruction of the vectors collected in $X$ up to a unitary transformation. Assuming that the camera is mounted on a mobile robot, we propose to fix the unitary transformation in accordance with the motion degrees of freedom of the robot.

Having reconstructed the topology of the imaging sensor allows doing 2D interpolation and therefore computing (approximated) directional derivatives and finding feature points using standard image processing techniques.
%
Then, considering for example that a camera has experienced from $t_1$ to $t_2$ a leftwards pan motion and from $t_3$ to $t_4$ an upwards tilt motion, where $t_i$ denote timestamps, allows computing two median optical-flows (or disparities), $v_1$ and $v_2$. The two flow vectors allow therefore setting the coordinates of a pixel location to be first horizontal, growing right, and the second to be vertical, growing down:
%
\begin{equation}
X_f= T X = [\hat v_1 \ \hat v_2]^{-1} X
\label{eq:rot_mirror}
\end{equation}
%
where $\hat v_1$ and $\hat v_2$ denote normalization to unit length of $v_1$ and $v_2$. Note that noise prevents perfect orthogonality, i.e. $v_1^T v_2 \ne 0$, in which case we rotate both vectors in opposing directions to meet orthogonality. Having $v_1^T v_2 = 0$ with nonzero $v_1$ and $v_2$, implies $|\det(T)| = 1$, where $\det(T)=-1$ indicates a mirroring effect found in the reconstructed topology.


