\documentclass[letterpaper]{llncs}
%\documentclass{article}
\usepackage{amsmath}
\usepackage{amsfonts}
\usepackage{amssymb}
\usepackage{graphicx}
%\usepackage[margin=3cm]{geometry}
\usepackage{array}
%\usepackage{amsthm}
\graphicspath{{./figs/setup/}{./figs/}{../1205_bmvc/images/}{../figs/}}


\newtheorem {mydef}{Definition}
\newtheorem {myprop}{Property}
\newtheorem {myobs}{Observation}
\newtheorem {myalg}{Algorithm}

%\usepackage{float}
%\floatstyle{ruled}
%\newfloat{algorithm}{tbp}{loa}
%\floatname{algorithm}{Algorithm}


 \setlength{\topmargin}{-10mm}
 \setlength{\textheight}{235mm}
 %\setlength{\topmargin}{0mm}
 \setlength{\headheight}{14pt}
 \setlength{\headsep}{10mm}
 \setlength{\textwidth}{150mm}
 \setlength{\footskip}{13mm}
 \setlength{\oddsidemargin}{5mm}
 \setlength{\evensidemargin}{5mm}



%\title{Lens Classification for Discrete Cameras}
\title{Lens Auto-Classification using a Featureless Methodology}

%
\titlerunning{Lens Classification}  % abbreviated title (for running head)
%                                     also used for the TOC unless
%                                     \toctitle is used
%
\author{R. Galego\inst{1} \and R. Ferreira\inst{1} \and A. Bernardino\inst{1} \and E. Grossmann\inst{2} \and J. Gaspar\inst{1}
}
%
\authorrunning{R. Galego et al.} % abbreviated author list (for running head)
%
%%%% list of authors for the TOC (use if author list has to be modified)
\tocauthor{R. Galego, R. Ferreira, A. Bernardino, E. Grossmann and J. Gaspar}
%
\institute{Institute for Systems and Robotics, IST/UTL, Portugal\\
\email{\{rgalego,ricardo,alex,jag\}@isr.ist.utl.pt}, \\ 
%WWW home page:\texttt{http://www.isr.ist.utl.pt}
\and
Intel Corp., Menlo Park, USA\\
\email{etienne@egdn.net} \\ 
}


% auxiliary commands
%
\newcommand{\R}[1] {\mathbb{R}^{#1}}
\newcommand{\Sphere}[1] {\mathbb{S}^{#1}}
\newcommand{\SO}[1] {\mathbb{SO}({#1})}

\newcommand{\Prob}[1] {\mathbb{P}\left[ #1 \right]}
\newcommand{\EV}[1] {\mathbb{E}\left[ #1 \right]}
\newcommand{\EVs}[1] {\mathbb{E}^2\left[ #1 \right]}

\newcommand{\p}[1] {\mathbf{#1}}
\def\pp {\p{p}}
\def\pq {\p{q}}
\def\acos {\operatorname{acos}}
\def\asin {\operatorname{asin}}
\def\Area {\operatorname{Area}}

\def\etal{\emph{et al. }}



\newcommand{\TODO}[1] {\textbf{*** #1  ***} }


%------------------------------------------------------------------------- 
% Document starts here
\begin{document}

\maketitle

\begin{abstract}
In this document we consider cameras having a cable of optic fibers forming a link between the image sensor (CCD) and a lens. We approach the problem of finding the lens type having estimated the sensor topology. We assume that we have a uniform density of fibers in the far-end of the cable.
\end{abstract}

%------------------------------------------------------------------------- 
\section{Introduction}

%Traditional calibration assumes imaging sensors formed by pixels precisely placed in a rectangular grid, as the most common artificial vision systems are like that...
%
%BUT non-regular topologies proved to be interesting: fast computations with little resources

%Traditional imaging sensors are composed by pixels precisely placed to form rectangular grids, and thus look like calibrated sensors for many practical purposes such as localizing local extrema, edges or corners.

Traditional imaging sensors are formed by pixels precisely placed in a rectangular grid, and thus look like calibrated sensors for many practical purposes such as localizing local extrema, edges or corners.
In contrast, the most common imaging sensors found in nature are the compound eyes,
%found in many crustaceans and, in particular, in arachnids or insects,
% such as flies or bees
collections of individual photo cells which clearly do not form rectangular grids, but are very effective for solving %e.g. navigation tasks
%and certainly inspire the design of artificial systems \cite{Neumann04,Di09}.
various tasks at hand and thus have inspired the design of many artificial systems.
%
Volkel \etal studied several types of eyes and discussed the miniaturization of imaging systems~\cite{Volkel03}.
%
Neumann \etal ~\cite{Neumann04} proposed a compound eye vision sensor for 3D ego motion computation.
%
Recently, Micro-Electro-Mechanical Systems fabrication technologies were applied to build artificial compound eyes on planar surfaces~\cite{Di09}.

%Compound eyes are the type of eyes most commonly seen in many crustaceans and, in particular, in arachnids or insects, such as flies or bees
%Being composed by a set of individual photo cells, they allow obtaining wide fields of view and detecting fast movements with few resources. % computational resources.
%%However compound eyes have a small resolution to avoid an overflow of the animals neural system with too much data.
%%
%In recent years the scientific community started to focus on this type of vision aiming to mimic its advantages.
%
%V\"{o}lkel \etal studied several types of eyes and their repercussions into electronic imaging and micro-optical designs~\cite{Volkel03}.

In other words, novel fabrication technologies allow creating sensors with pixel arrangements (topologies) tuned for the tasks at hand.
%Many novel sensors will not have pixels forming regular rectangular grids.
%In some cases sensor fabrication may benefit of allowing the emergence of sensor topologies instead of rigidly imposing them.
%This precludes using traditional calibration methodologies~\cite{Agapito99,Zhang99,Sinha04}.
%In many cases this will preclude using traditional calibration methodologies~\cite{Agapito99,Zhang99,Sinha04}.
In the cases where the sensor topology is not a rectangular grid using traditional calibration methodologies~\cite{Agapito99,Zhang99,Sinha04} will not be possible.
%
Hence, the question arising here is: how to calibrate sensors with unknown topologies?
%
%Assuming that the sensors are mounted on mobile robots, the previous question can be reformulated as: can we calibrate an unknown topology sensor just with the data acquired by the sensor? 
%
In the case that the sensors are mounted on mobile robots the question can be restated as: can we calibrate an unknown topology of a moving sensor just with the data acquired by the sensor? 

Pierce and Kuipers %pioneered topological calibration by proposing methodologies and doing experiments showing 
have shown
that it is possible to reconstruct the topology of a group of sensors just by knowing their output~\cite{Kuipers97}.
%Pierce and Kuipers introduced the notion that is possible to reconstruct the topology of a group of sensors just by knowing their output~\cite{Kuipers97}.
They use natural %generic 
properties of an agent's world in order to infer the structure of its sensors.
%In particular they use two types of metrics: one based on the premise that usually adjacent sensors have similar values, a second metric is based on the frequency distributions of the sensors, since similar sensors should have the same frequency distributions.
%
Olsson \etal improved the methodologies introduced by Pierce and Kuipers by adding information distances and,
%in the case of de-scrambling a $20\times20$ pixels sensor,
%in particular, Hamming metrics~\cite{Olsson04}.
%More recently they compare the effects of different types of distance metrics~\cite{Olsson06}. 
in particular, Hamming metrics~\cite{Olsson04,Olsson06}.
They compute the position of several sensors of a Sony Aibo robot, which has, among other sensors, one camera sub-sampled to $8\times 8$ pixels.
%
%\TODO{Rewrite: Despite of the results they only use a $8\times8$ pixels camera, and they do not give information how does the information distance is related with the metric distance}.
%
%It was also stated by Hyvarinen \etal that a neurons (feature detectors) \emph{topography} can be obtained with a simple modification of the model of independent subspace analysis (ISA).
%
Hyvarinen \etal shown that imaging natural scenes allow defining a \emph{neuronal topography} %based on a simple modification of the model of 
using Independent Subspace Analysis~\cite{nis09}. % (ISA)~\cite{nis09}. %ISA is a generalization of a statistical generative model, the Independent Component Analysis, whose estimation boils down to sparse coding~\cite{nis09}.
%
Recently, Grossmann \etal~\cite{Grossmann10} proposed a method for calibrating a central imaging sensor based on a number of photocells. They need to know (estimate) a priori a function curve relating correlation (or information-distance) and distance-angles. % in order their algorithm works. 
Their algorithm has been tested on a small set of pixels (photocells), about one hundred, as otherwise the computation time and memory would be too large.

In this work we want to do auto-calibration of central sensors with a number of pixels orders of magnitude larger than \cite{Olsson06,Grossmann10}.
%\TODO{cite google uses the same tecnic for 17M points}
We approach the computational complexity with Multi Dimensional Scaling (MDS) like algorithms. A relatively old but very effective in the presence of noise free data is the Classical MDS~\cite{MDS}, based on Euclidean distances. Its goal is to find a representation of a data set on a given dimensionality from the knowledge of all interpoint distances.
Several new algorithms evolved from MDS, such as Isomap~\cite{Isomap}, where geodesic distances induced by a neighborhood graph are used instead of Euclidean distances. More recently Landmark Isomap~\cite{Landmark} was introduced, which uses only a subset of the all-to-all distances used in Isomap and has been proven to work on large scale datasets %(up to 18 million data points)~\cite{Google08}.
(millions of data points)~\cite{Google08} and constitute therefore a promising research direction. 
%approach to topological calibration.

%Another branch of the multi dimension scaling algorithms is the Self Organizing Maps (SOM), also known as Kohonen maps \cite{Kohonen81}. SOM are networks that use unsupervised training to preserve the topological properties of the input space. These maps need to have prior information of the topology, usually hexagonal or rectangular grid, to use in a neighborhood function. One characteristic of SOM is that it can be feed by different types of input metric and non metric.

The structure of the paper is the following:
in Sec.2 we study a simple black and white scenario and show there is a linear relationship between the correlation of the time series acquired by pairs of pixels and the inter-pixel angle;
in Sec.3 we describe Landmark-Isomap applied to topological calibration and propose
%a rotation and/or mirroring correction methodology;
a methodology for choosing a coordinate frame for the imaging sensor;
in Sec.4 we show some experimental results, and finally in Sec.5 we draw some conclusions.

%\section{Introduction}
%
%Assuming that we have a sensor with a known density distribution in A, is it possible to know what type of lens do we have at the other end of the sensor based on the topology of the raxels? 


%\section{Camera Model}
%
%Our camera model is composed by a CCD camera as it can be seen in figure~\ref{fig:setup} marked with a $A$, in front of the camera is a cable of optic fibers ($B$), and in the front of the the fibers is a lens ($C$). The fibers of the optic fiber cable are randomly mixed inside the cable, meaning that a fiber has a position $(u_e,v_e)$ at one end ($E$) and at the other end ($D$) has a different position $(u_d,v_d)$ i.e. $(u_e,v_e)\neq (u_d,v_d)$.
%
%\begin{figure}
%\centering
%\includegraphics[width=12.0cm]{setup_v1.eps} 
%\caption{An overview of the setup, and some examples of the images acquired in each step}
%\label{fig:setup}
%\end{figure}
%%
%%sensor with a known density distribution in A,
%
%Central cameras can b
%
%\begin{eqnarray}
%\Omega=h(r/l) \\
%r/l=h^{-1} (\Omega)
%\end{eqnarray}
%
%In our work we assume that we have three different lens: 
%consider that r is the radial distance from the center of an imaging sensor, $l$ is the focal length and $\Omega$ is the angle between the principal axis and the incoming ray.
%In the following we assume  that we have three different lens, marked in the figure as $c$: 
%
%%\begin{enumerate} 
%%	\item  perspective lens $\Omega =atan(r/l)$
%%	\item equidistance projection lens, $\Omega =r/l$
%%	\item orthogonal projection lens, $\Omega =asin(r/l)$
%%\end{enumerate}
%
%\begin{description}
	%\item[Case 1] perspective lens, $\Omega =atan(r/l)$ 
	%\item[Case 2] equidistance projection lens, $\Omega =r/l$
	%\item[Case 3] orthogonal projection lens, $\Omega =asin(r/l)$
%\end{description}
%
%From now on the focal length $l$ will be not considered since it is a constant which cannot be
%observed.
%
%Grossberg and Nayar\cite{Raxels} defined the concept of raxels, raxels is a mathematical abstraction of the position of the light sensor. Instead of the real position of the sensor, raxels are assumed to be along the direction of the chief ray associated to the sensor, they can be catheterized as a $3D$ position and a direction vector, as it can be seen in Fig.\ref{fig:setup}(d).





\section{Camera Model}

%Our camera model is composed by a CCD camera as it can be seen in figure~\ref{fig:setup} marked with the letter $A$, in front of the camera is a cable of optic fibers ($B$), and in the front of the the fibers is a lens ($C$). The fibers of the optic fiber cable are randomly mixed inside the cable, meaning that a fiber has a position $(u_e,v_e)$ at one end ($E$) and at the other end ($D$) has a different position $(u_d,v_d)$ i.e. $(u_e,v_e)\neq (u_d,v_d)$.

%\begin{figure*}
%\centering
%\includegraphics[width=10.0cm]{setup_v1.eps} 
%\caption{An overview of the setup, and some examples of the images acquired in each step}
%\label{fig:setup}
%\end{figure*}
%%


%sensor with a known density distribution in A,

Discrete central cameras, as conventional (standard) cameras, are described geometrically by the pin-hole projection model. Differently from standard cameras, discrete cameras are simply composed of collections of pixels organized as pencils of lines with unknown topologies.
%
%We assume to have a discrete central camera. Central cameras can be defined as a collection of pixels, where the integration area of each pixel collapses to a point. 

Grossberg and Nayar\cite{Raxels} defined the concept of raxels, which is a mathematical abstraction of the position of the light sensor. Instead of the real position of the sensor, raxels are assumed to be along the direction of the chief ray associated to the sensor, they can be characterized as a $3D$ position, $p$, and a direction vector, $q$, as it can be seen in Fig.\ref{fig:setup}(d).
%
However since we are working with central cameras, one considers that all light rays converge to the same point, thus, making $p_i=p_j$ for any pair of raxels. Therefore, one can ignore the raxels position, since the only useful information is contained in the direction vector $q$. As a direction vector, $q$, one can assume that all the vectors have the same norm, this assumption removes one degree of freedom, which allow us to represent $q$ with only two angles, $(\Omega,\mu)$.

Traditionally the coordinate system of a camera sensor is represented by $u_i$ and $v_i$, which are the ith pixel position along horizontal and vertical grid. Here we use a polar coordinate system of $[r\ \mu]$, where $r_i=\sqrt{(u_i-u_0)^2+(v_i-v_0)^2}$, $\mu_i =\acos (u_i/v_i)$, and $[u_0\ v_0]$ is the principal point. %position of the center of the sensor.


In this work we assume that the discrete camera geometric model can be characterized by an unknown radial function $h$. This function links the angle at which a light ray (raxel) hits the camera lens with the imaged point (pixel coordinates), 
%In our work we assume that we have three different lens, where every lens has know function $h^{-1}$. This function links the angle at which a light ray hits the camera lens with the imaged point (pixel coordinates),  
%
\begin{equation}
\Omega=h(r/l),% \\
%r/l=h^{-1} (\Omega)
\end{equation}
%
considering that $r$ is the radial distance, in pixels, from the center of an imaging sensor, $l$ is the focal length and $\Omega$ is the angle between the principal axis and the incoming ray, as it can be seen in figure~\ref{fig:setup} (d), note that the $\mu$ is the same as in pixels coordinates, since the lens transformation only affect the radius.

In the following we assume  that we have three different lenses\cite{Kannala06}: 
%
%\begin{enumerate} 
%	\item  perspective lens $\Omega =atan(r/l)$
%	\item equidistance projection lens, $\Omega =r/l$
%	\item orthogonal projection lens, $\Omega =asin(r/l)$
%\end{enumerate}
%
%\begin{enumerate}
%
%
\begin{eqnarray}
\Omega =atan(r/l) \label{eq:presp} \hspace{3.1cm} \text{perspective lens,}\\
%
%\end{equation}
%
%\textbf{2} equidistance projection lens,
%\begin{equation}
\Omega =r/l \hspace{2.3cm} \text{equidistance projection lens,}\label{eq:equi}\\
%
%\end{equation}
%	
%\textbf{3} orthogonal projection lens,
%\begin{equation}
\Omega =asin(r/l)  \hspace{1.6cm} \text{orthogonal projection lens.} 
%\Omega =asin(r/l).
\label{eq:ortho}
\end{eqnarray}
%\end{enumerate}
%
From now on the focal length $l$ will be not considered since it is a constant that will not have impact on the differentiation of a lens type.

In order to classify a lens mounted on a camera one as to follow the next steps: i)  calibrate topologically a sensor; ii) get the marginal density of the topology along $\mu$; iii) find the lens that has the closest marginal density function to  the marginal density function of the topology. 



\section{Discrete Camera Model}



\begin{figure*}[!ht]
\centering
\begin{tabular}{cccc}
%\includegraphics[width=1.4\columnwidth]{setup_v1.eps} 
\includegraphics[height=4.5cm]{dcam_diagr.eps} &
\hspace{0.5cm}
\includegraphics[height=4.5cm]{dcam_coord_frames.eps} &
\hspace{1cm}
%\includegraphics[height=4.5cm]{dcam_imgs.eps} &
\includegraphics[height=4.5cm]{dcam_imgs_grey.eps} &
%\includegraphics[width=.4\columnwidth]{raxel_v2.eps} \\
%\includegraphics[height=4.5cm]{raxel_v2.eps} \\
\hspace{1cm}
\includegraphics[height=4.5cm]{dcam_raxels.eps} \\
(a) Camera and basis & (b) Coordinate frames & (c) Images at D and E & (d) Projection model \\
\end{tabular}
%\caption{An overview of the setup, and some examples of the images acquired in each step}
\caption{Model of a discrete camera mounted on a pan-tilt basis. The optic-fiber bundle, points E to D in (a), twists the input image (d). Vectors $v_1$ and $v_2$ allow computing a unity transform to obtain $\{4\}$ (b). Projection model and raxels notation (d).}
%\caption{Characterizing raxels.}
\label{fig:setup}
\end{figure*}


%%In this work we have two types of camera, one that is a regular CCD with a lens in front. 
%
%%A discrete camera is defined as a collection of pixels,
%A discrete camera is a set of pixels, disposed in a general topology.
%%organized in a general topology. 
%
%A discrete camera is composed by a collection of photocells characterized geometrically by a pencil of 3D optical rays.
%
Discrete cameras, as conventional (standard) cameras, are described geometrically by the pin-hole projection model. Differently from standard cameras, discrete cameras are simply composed of collections of pixels organized as pencils of lines with unknown topologies.
%
%In this work we assume point pixels, meaning that the integration area of each pixel collapses to a point.   
%The camera topology can be defined as the collection of pixels, which is unknown and is to be found.


Figure~\ref{fig:setup} shows a model of a discrete camera.
%
The discrete camera is composed by
%
one conventional CCD camera (see Fig.~\ref{fig:setup}(a) label $A$),
%
one cable of optic fibers mounted in front of the camera ($B$),
%
and one extra lens mounted in front of the fibers ($C$).
% 
The fibers of the optic fiber cable are randomly mixed inside the cable, meaning that a fiber has a position $(u_e,v_e)$ at one end ($E$) and at the other end ($D$) has a different position $(u_d,v_d)$, i.e. $(u_e,v_e)\neq (u_d,v_d)$. 
%Note that in this case the only information available are a set of pixel streams, $f_i$, a time series of brightness values captured by $i$th pixel.


Grossberg and Nayar\cite{Raxels} have introduced the concept of raxel to allow representing more general cameras. A raxel is in simple terms an abstraction of the position of a light (punctual) sensor. Instead of representing the real position of the light (punctual) sensor, a raxel is just representing the direction of the chief ray associated to the sensor. A raxel is characterized as a $3D$ position and a direction vector. %, as it can be seen in Fig.\ref{fig:setup}(d).


In this work we consider central discrete cameras. These cameras are represented as collections of raxels. Since the cameras are central, the $3D$ position associated to each raxel can be the same for all raxels. Raxels can therefore be represented as vectors on the unit sphere $\p{x_i}\in\Sphere{2}$.
%
%
%In this work we assume that the discrete camera geometric model can be characterized by an unknown radial function $h$. This function links the angle at which a light ray (raxel) hits the camera lens with the imaged point (pixel coordinates),  
%%
%\begin{equation}
%\Omega=h(r/l) 
%%r/l=h^{-1} (\Omega)
%\end{equation}
%%
%where $r$ is the radial distance (pixels), from the center of the imaging sensor, $l$ is the focal length and $\Omega$ is the angle between the principal axis and the incoming ray (see Fig.~\ref{fig:setup}(d)).
%where $r$ denotes the radial distance from the center of the imaging sensor to the point where the raxel crosses the image plane (pixels), $l$ is the focal length and $\Omega$ is the angle between the principal axis and the incoming ray (see Fig.~\ref{fig:setup}(d)).
%
Each raxel can also be characterized by two values, $(\Omega_i, \mu_i)$.
$\Omega_i$ is the angle between the principal axis and the incoming ray (see Fig.~\ref{fig:setup}(d)).
%
The azimuthal angle $\mu_i$ characterizes both the angular location of an imaged point and the plane containing a raxel.
%is the same as in pixels coordinates, since the lens transformation only affect the radius. This is true in any type of central camera.


In uncalibrated discrete cameras is not possible to define corner points or image lines since the topology is unknown. The only information available for camera calibration
are a set of pixel streams, $\{ f_i \}$, where each pixel-stream $f_i$ corresponds to a time series of brightness values captured by $i^{th}$ photocell (pixel).
%
%In order to build a topology one has to find a way how use the available information. % and transform it in to a distance information.
Galego~\etal\cite{Galego13} have shown that the normalized cross correlation of two pixels streams, $C(f_i, f_j)$, is affine to the angular distance of pixels, $d(\p{x}_i,\p{x}_j)$, when observing a circular object.
%
Having all-to-all raxel angular distances, one can use %the multidimensional scaling algorithm 
MDS to estimate the full topology. 
%
In calibration methodology can therefore be summarized as
%
(i) receive a set of pixels streams,
(ii) compute the cross correlation between all pixels streams,
(iii) convert all correlation values into angular distances, and finally
(iv) estimate the topology using the distances previously computed.
%
See Fig.~\ref{fig:diagram}.

\input{sec5_fig}




\section{Auto-Calibration Methodology}
\label{sec:AutoCalibrationMethodology}
%For the particularly simple scenery described in section \ref{sec:TopologyOfPixelStreams}, it is clear that for correlation values above 0.5 the look up table needed in \cite{Grossmann10} can be replaced by a single slope parameter dependent on the size of the observed hubcap. We propose to further drop this requirement for small sensors by embedding the locations in a plane up to a scale factor. The proposed method consists of directly converting the obtained correlation values to distance values and then applying the landmark Isomap algorithm \cite{Landmark} to obtain the reconstruction. Hence we will consider that the distance between two pixels whose correlation is above $0.5$ is given by $d(\p{x}_i, \p{x}_j) = 1-C(f_i,f_j)$. In the following we provide a short description of the Landmark Isomap algorithm. 

%\paragraph{MDS and Landmark Isomap -}
The classical Multiple Dimensional Scaling (MDS) algorithm~\cite{MDS} provides a simple way of embedding a set of points in Euclidean space given their inter-distances. 
%%The first step is recognizing that the squared distance function is a linear transformation of the inner product between points
%
%\begin{equation}
%d^2(\p{x}_i,\p{x}_j) =\langle \p{x}_i-\p{x}_j, \p{x}_i-\p{x}_j \rangle 
 %=\langle \p{x}_i, \p{x}_i \rangle-2\langle \p{x}_i, \p{x}_j \rangle+\langle \p{x}_j, \p{x}_j \rangle
%\end{equation}
%
%Collecting all squared distances in a matrix $D^2= [d^2(\p{x}_i,\p{x}_j)]$ this can be transformed to a matrix of inner products by inverting the previous linear relation and forcing the resulting embedding to have zero mean \cite{Dattorro10}. More precisely, a matrix of inner products $G$ is obtained from $D^2$ through the transformation $G= -J D^2 J/2$, where $J=(I-1/n)$, $I$ is the $n\times n$ identity matrix and $n$ is the number of the elements of the data. Next, one observes that if the desired point embedding is collected in a matrix $X= [\p{x}_1\ \p{x}_2\ \cdots \ \p{x}_n]$, then 
%\begin{equation}
%G=\left [\begin{array}{ccc} \langle \p{x}_1, \p{x}_1\rangle & \hdots & \langle \p{x}_1, \p{x}_n\rangle \\
 %\vdots & \ddots & \vdots \\ 
  %\langle \p{x}_n, \p{x}_1\rangle & \hdots & \langle \p{x}_n, \p{x}_n\rangle \end{array}
 %\right] = X^TX.
%\end{equation}
%%
%$X$ can thus be obtained (reconstructed) up to a unitary transformation using an SVD decomposition, as $G=X^TX=U\Sigma U^T=U\sqrt{ \Sigma} \sqrt{\Sigma}U^T$,
%and thus $X^T=U\sqrt{\Sigma}$.
%
%\begin{myprop} 
%If distances are multiplied by a scale factor, $d_1=\alpha d$, then the resulting topology is scaled by the same factor, $X_1=\alpha X$ .
%\label{prop:mds_scale}
%\end{myprop}

%\begin{proof}
%%Demo: 
%Suppose all distances are affected by a gain $\alpha$, $d_1=\alpha d$, then the matrix of squared distances will be $D^2_1=\alpha^2 D^2$. The inner product matrix will have a gain of $\alpha^2$ since $G_1=\alpha^2JD^2J/2$ which will make the SVD of matrix $G_1=U(\alpha^2\Sigma)U^T$. The solution of the algorithm will then be $X_1^T=U\sqrt{\alpha^2\Sigma}$. Since $X^T=U\sqrt{\Sigma}$  one concludes that $X_1=X\alpha$ which proves that a gain in the distances changes the topology of the sensor only by a scale factor as required.
%\end{proof}
%%$\Box$


%As noted by Tenenbaum \etal \cite{Isomap}, 
The classical MDS works well when the distances are Euclidean and when the structures are linear, however, when the manifolds are nonlinear, the classical MDS fails to detect the true dimensionality of the data set. 
%
Isomap is built on classical MDS but instead of using Euclidean distances it uses an approximation of geodesic distances~\cite{Isomap}. These geodesic distance approximations are defined as a series of hops between neighboring points in the Euclidean space using a shortest path graph algorithm such as Dijkstra's.

Isomap has two computational bottlenecks, namely the memory and time complexity of computing an all-to-all distance matrix, $n\times n$ for a $n$ pixels camera, and computing its eigenvalues.
Landmark Isomap improves both inefficiencies~\cite{Landmark}.  Instead of using all the data points Landmark Isomap proposes using just $k$ randomly selected points (landmarks) with $k \ll n$. The embedding is done the same way as in MDS but using only the $k$ landmarks. 
%The distance matrix, $D$ is now just a $k\times n$ matrix. After embedding the landmarks, $K=\sqrt{\Sigma}U^T$, where $K$ collects the $k$ reconstructed landmark locations, one can embed the %$n-y$ 
%remaining points:
%
\begin{equation}
X=\frac{1}{2}K^*(\overline{D_k}-D_k)
\label{eq:iso_other_pts}
\end{equation}   
%
where $K^*=[u_1^T/\sqrt{\Sigma_1}\ \hdots \ u_s^T/\sqrt{\Sigma_s}]^T$ is the pseudo-inverse transpose of $K$, $s$ is the size of the dimensionality space, $D_k$ is the distance matrix from the landmarks to the complete data, and $\overline{D_k}$ is the mean of the columns of $D_k$.  

In our particular case, this algorithm is used to provide a pixel embedding given the inter-pixel distances estimated from the pixel stream correlations.

%\paragraph{Choosing the Imaging Coordinate System - }
%
%As referred, MDS (and derived methods) provide a reconstruction of the vectors collected in $X$ up to a unitary transformation. Assuming that the camera is mounted on a mobile robot, we propose to fix the unitary transformation in accordance with the motion degrees of freedom of the robot.
%
%Having reconstructed the topology of the imaging sensor allows doing 2D interpolation and therefore computing (approximated) directional derivatives and finding feature points using standard image processing techniques.
%%
%Then, considering for example that a camera has experienced from $t_1$ to $t_2$ a leftwards pan motion and from $t_3$ to $t_4$ an upwards tilt motion, where $t_i$ denote timestamps, allows computing two median optical-flows (or disparities), $v_1$ and $v_2$. The two flow vectors allow therefore setting the coordinates of a pixel location to be first horizontal, growing right, and the second to be vertical, growing down:
%%
%\begin{equation}
%X_f= T X = [\hat v_1 \ \hat v_2]^{-1} X
%\label{eq:rot_mirror}
%\end{equation}
%%
%where $\hat v_1$ and $\hat v_2$ denote normalization to unit length of $v_1$ and $v_2$. Note that noise prevents perfect orthogonality, i.e. $v_1^T v_2 \ne 0$, in which case we rotate both vectors in opposing directions to meet orthogonality. Having $v_1^T v_2 = 0$ with nonzero $v_1$ and $v_2$, implies $|\det(T)| = 1$, where $\det(T)=-1$ indicates a mirroring effect found in the reconstructed topology.


\paragraph{Calibration Methodology -}

Summarizing the previous sections, estimating and embedding the topology of a central imaging sensor involves acquiring a set of images observing a bright circle faraway, e.g. the full moon. Since the correlation of pixel-streams, $C(f_i,f_j)$ is invariant to shuffling of the time-series (as long as both time series are affected by the same shuffling), these images can be acquired either as a continuous sequence (i.e. a video) or as discrete individual images. In the end one wants to obtain the embedded pixel locations, $X_f= [\p{x}_1\ \p{x}_2\ \cdots \p{x}_N]$.
%
The required steps are the following:
%
(i) Data binarization using a fixed threshold such that each pixel stream value is either $1$ or $-1$.
%
(ii) Computing the normalized correlation between all the pixel-streams using Eq.\ref{eq:correlation_simplified}.
%
(iii) Converting the inter-pixel correlations into distances, using the linear transformation $d(\p{x}_i,\p{x}_j)=1-C(f_i,f_j)$ (based on properties \ref{prop:corr_linearity} and \ref{prop:mds_scale}).
%
(iv) Using Landmark Isomap to compute the topology of the sensor. 
%
(v) (optional in case an external reference frame is available) Choosing a coordinate system for the camera based on the supporting robot motion (Eq.\ref{eq:rot_mirror}). 


\section{Mirror effect}

As referred, MDS (and derived methods) provide a reconstruction of the vectors collected in $X$ up to a unitary transformation. Assuming that the camera is mounted on a mobile robot, we propose to fix the unitary transformation in accordance with the motion degrees of freedom of the robot.

Having reconstructed the topology of the imaging sensor allows doing 2D interpolation and therefore computing (approximated) directional derivatives and finding feature points using standard image processing techniques.
%
Then, considering for example that a camera has experienced from $t_1$ to $t_2$ a leftwards pan motion and from $t_3$ to $t_4$ an upwards tilt motion, where $t_i$ denote timestamps, allows computing two median optical-flows (or disparities), $v_1$ and $v_2$. The two flow vectors allow therefore setting the coordinates of a pixel location to be first horizontal, growing right, and the second to be vertical, growing down:
%
\begin{equation}
X_f= T X = [\hat v_1 \ \hat v_2]^{-1} X
\label{eq:rot_mirror}
\end{equation}
%
where $\hat v_1$ and $\hat v_2$ denote normalization to unit length of $v_1$ and $v_2$. Note that noise prevents perfect orthogonality, i.e. $v_1^T v_2 \ne 0$, in which case we rotate both vectors in opposing directions to meet orthogonality. Having $v_1^T v_2 = 0$ with nonzero $v_1$ and $v_2$, implies $|\det(T)| = 1$, where $\det(T)=-1$ indicates a mirroring effect found in the reconstructed topology.





%\section{Introduction}

%Traditional calibration assumes imaging sensors formed by pixels precisely placed in a rectangular grid, as the most common artificial vision systems are like that...
%
%BUT non-regular topologies proved to be interesting: fast computations with little resources

%Traditional imaging sensors are composed by pixels precisely placed to form rectangular grids, and thus look like calibrated sensors for many practical purposes such as localizing local extrema, edges or corners.

Traditional imaging sensors are formed by pixels precisely placed in a rectangular grid, and thus look like calibrated sensors for many practical purposes such as localizing local extrema, edges or corners.
In contrast, the most common imaging sensors found in nature are the compound eyes,
%found in many crustaceans and, in particular, in arachnids or insects,
% such as flies or bees
collections of individual photo cells which clearly do not form rectangular grids, but are very effective for solving %e.g. navigation tasks
%and certainly inspire the design of artificial systems \cite{Neumann04,Di09}.
various tasks at hand and thus have inspired the design of many artificial systems.
%
Volkel \etal studied several types of eyes and discussed the miniaturization of imaging systems~\cite{Volkel03}.
%
Neumann \etal ~\cite{Neumann04} proposed a compound eye vision sensor for 3D ego motion computation.
%
Recently, Micro-Electro-Mechanical Systems fabrication technologies were applied to build artificial compound eyes on planar surfaces~\cite{Di09}.

%Compound eyes are the type of eyes most commonly seen in many crustaceans and, in particular, in arachnids or insects, such as flies or bees
%Being composed by a set of individual photo cells, they allow obtaining wide fields of view and detecting fast movements with few resources. % computational resources.
%%However compound eyes have a small resolution to avoid an overflow of the animals neural system with too much data.
%%
%In recent years the scientific community started to focus on this type of vision aiming to mimic its advantages.
%
%V\"{o}lkel \etal studied several types of eyes and their repercussions into electronic imaging and micro-optical designs~\cite{Volkel03}.

In other words, novel fabrication technologies allow creating sensors with pixel arrangements (topologies) tuned for the tasks at hand.
%Many novel sensors will not have pixels forming regular rectangular grids.
%In some cases sensor fabrication may benefit of allowing the emergence of sensor topologies instead of rigidly imposing them.
%This precludes using traditional calibration methodologies~\cite{Agapito99,Zhang99,Sinha04}.
%In many cases this will preclude using traditional calibration methodologies~\cite{Agapito99,Zhang99,Sinha04}.
In the cases where the sensor topology is not a rectangular grid using traditional calibration methodologies~\cite{Agapito99,Zhang99,Sinha04} will not be possible.
%
Hence, the question arising here is: how to calibrate sensors with unknown topologies?
%
%Assuming that the sensors are mounted on mobile robots, the previous question can be reformulated as: can we calibrate an unknown topology sensor just with the data acquired by the sensor? 
%
In the case that the sensors are mounted on mobile robots the question can be restated as: can we calibrate an unknown topology of a moving sensor just with the data acquired by the sensor? 

Pierce and Kuipers %pioneered topological calibration by proposing methodologies and doing experiments showing 
have shown
that it is possible to reconstruct the topology of a group of sensors just by knowing their output~\cite{Kuipers97}.
%Pierce and Kuipers introduced the notion that is possible to reconstruct the topology of a group of sensors just by knowing their output~\cite{Kuipers97}.
They use natural %generic 
properties of an agent's world in order to infer the structure of its sensors.
%In particular they use two types of metrics: one based on the premise that usually adjacent sensors have similar values, a second metric is based on the frequency distributions of the sensors, since similar sensors should have the same frequency distributions.
%
Olsson \etal improved the methodologies introduced by Pierce and Kuipers by adding information distances and,
%in the case of de-scrambling a $20\times20$ pixels sensor,
%in particular, Hamming metrics~\cite{Olsson04}.
%More recently they compare the effects of different types of distance metrics~\cite{Olsson06}. 
in particular, Hamming metrics~\cite{Olsson04,Olsson06}.
They compute the position of several sensors of a Sony Aibo robot, which has, among other sensors, one camera sub-sampled to $8\times 8$ pixels.
%
%\TODO{Rewrite: Despite of the results they only use a $8\times8$ pixels camera, and they do not give information how does the information distance is related with the metric distance}.
%
%It was also stated by Hyvarinen \etal that a neurons (feature detectors) \emph{topography} can be obtained with a simple modification of the model of independent subspace analysis (ISA).
%
Hyvarinen \etal shown that imaging natural scenes allow defining a \emph{neuronal topography} %based on a simple modification of the model of 
using Independent Subspace Analysis~\cite{nis09}. % (ISA)~\cite{nis09}. %ISA is a generalization of a statistical generative model, the Independent Component Analysis, whose estimation boils down to sparse coding~\cite{nis09}.
%
Recently, Grossmann \etal~\cite{Grossmann10} proposed a method for calibrating a central imaging sensor based on a number of photocells. They need to know (estimate) a priori a function curve relating correlation (or information-distance) and distance-angles. % in order their algorithm works. 
Their algorithm has been tested on a small set of pixels (photocells), about one hundred, as otherwise the computation time and memory would be too large.

In this work we want to do auto-calibration of central sensors with a number of pixels orders of magnitude larger than \cite{Olsson06,Grossmann10}.
%\TODO{cite google uses the same tecnic for 17M points}
We approach the computational complexity with Multi Dimensional Scaling (MDS) like algorithms. A relatively old but very effective in the presence of noise free data is the Classical MDS~\cite{MDS}, based on Euclidean distances. Its goal is to find a representation of a data set on a given dimensionality from the knowledge of all interpoint distances.
Several new algorithms evolved from MDS, such as Isomap~\cite{Isomap}, where geodesic distances induced by a neighborhood graph are used instead of Euclidean distances. More recently Landmark Isomap~\cite{Landmark} was introduced, which uses only a subset of the all-to-all distances used in Isomap and has been proven to work on large scale datasets %(up to 18 million data points)~\cite{Google08}.
(millions of data points)~\cite{Google08} and constitute therefore a promising research direction. 
%approach to topological calibration.

%Another branch of the multi dimension scaling algorithms is the Self Organizing Maps (SOM), also known as Kohonen maps \cite{Kohonen81}. SOM are networks that use unsupervised training to preserve the topological properties of the input space. These maps need to have prior information of the topology, usually hexagonal or rectangular grid, to use in a neighborhood function. One characteristic of SOM is that it can be feed by different types of input metric and non metric.

The structure of the paper is the following:
in Sec.2 we study a simple black and white scenario and show there is a linear relationship between the correlation of the time series acquired by pairs of pixels and the inter-pixel angle;
in Sec.3 we describe Landmark-Isomap applied to topological calibration and propose
%a rotation and/or mirroring correction methodology;
a methodology for choosing a coordinate frame for the imaging sensor;
in Sec.4 we show some experimental results, and finally in Sec.5 we draw some conclusions.

%\input{sec1_related}
%\input{sec2_calibr}
%\input{sec3_corr}
\section{Results}

%\subsection{Find sensor topology}
%Figure~\ref{fig:results1} shows ...
%\begin{figure}
\centering
\begin{tabular}{cccc}
\includegraphics[width=2.0cm]{moon_view_1.eps} & 
\includegraphics[width=2.0cm]{moon_view_4.eps} & 
\includegraphics[width=2.0cm]{moon_view_4.eps} &
\includegraphics[width=2.0cm]{moon_view_2.eps} \\
%\includegraphics[width=2.0cm]{empty.eps} & 
%\includegraphics[width=2.0cm]{empty.eps} & 
%\includegraphics[width=2.0cm]{empty.eps} &
%\includegraphics[width=2.0cm]{empty.eps} \\
(a) moon view &
(b) moon view &
(c) moon view &
(d) moon view \\
 frame 890 &
 frame 2703 &
 frame 2703 &
 frame 5215 %\\
\vspace{3mm} \\
\end{tabular}
%
%\begin{tabular}{cc}
%\includegraphics[width=3cm]{moon_unrot.eps} & 
%\includegraphics[width=3cm]{moon_rot.eps} \\
%(d) Topology result & (e) Topology arrange  %\\
%\vspace{3mm} \\
%\end{tabular}
%
\begin{tabular}{ccc}
\includegraphics[height=3.5cm]{topology_32_lines_unrot.eps} & 
\includegraphics[height=3cm]{topology_32_lines_rot.eps} &
\includegraphics[height=2cm]{topology_32_lines_crop_rot.eps} \\
(d) Topology result & (e) Rotated  & (f) Zoomed crop  %\\
\vspace{3mm} \\
\end{tabular}
%
\caption{
Topology using minimum bounding box (note: 14784 frames).
}
\label{fig:results1}
\end{figure}


%\subsection{Application of the topology finding}

%Nikon D5000, selected a central 100x100 pixels region (why? non-optimized Dijkstra / matlab?) Known topology, simple rectangular grid (topology errors clear to show locally and globally).
% ^^ & Calibrated camera (no need?)

In order to test the proposed topology estimation methodology two experiments have been conducted using a Nikon D5000 camera in video mode, selecting just a central region of 100x100 pixels, and thus having the ground truth of a sensor composed by square pixels forming a regular square grid.
%Data acquisition was performed at 24fps, about ten minutes for each experiment, while panning, tilting and rolling the camera.

%1) perfect conditions: video of the moon in a full moon night, approximately 8min video, 15k images (14784 images); pan tilt (and roll? roll does not influence as the full moon is very close to a circle...)
%
%2) non perfect conditions, hand held camera recording a video in the campus garden and car park; pan tilt and roll, and some translation; binarization helps when using short sequences (Grossmann10 and plot in sec2)


%--- Exp1: shuffling of pixels: no particular order of pixel-streams - in practice verified that despite the random selection of landmarks, the reconstructions are very similar \TODO{(Procrustes error less than ....)}

%\begin{figure}
%\centering
%\begin{tabular}{ccc}
%\includegraphics[width=3cm]{face_case_35.eps} & 
%\multicolumn{2}{c}{ \includegraphics[width=5cm]{face_shuffle.eps} }\\
%(a) Test image & \multicolumn{2}{c}{(b) $2 \times 4$ shuffling } %\\
%\vspace{3mm} \\
%\includegraphics[width=3cm]{face_case_32.eps} & 
%\includegraphics[width=3cm]{face_case_34.eps} & 
%\includegraphics[width=3cm]{face_case_30.eps} \\
%(c) $2 \times 4$ shuffling &
%(d) $10 \times 10$ shuffling &
%(e) $100 \times 100$ shuffling %\\
%\vspace{3mm} \\
%\includegraphics[width=3cm]{topology_file_res02_case_32.EPS} & \includegraphics[width=3cm]{topology_file_res02_case_34.EPS} & \includegraphics[width=3cm]{topology_file_res02_case_30.EPS} \\
%(f) Reconstruction after & (g) Reconstruction after & (h) Reconstruction after \\
% $2 \times 4$ shuffling &
% $10 \times 10$ shuffling &
% $100 \times 100$ shuffling %\\
%\vspace{3mm} \\
%\end{tabular}
%\caption{
%Topology estimation applied to the reconstruction of a shuffled $100 \times 100$ sensor.
%%Topology estimation applied to reordering a shuffled sensor.
%%Topology estimation applied to the reconstruction of an image.
%Test image before shuffling (a).
%The pixels of the sensor are shuffled in (i) $2 \times 4$ blocks, (ii) $10 \times 10$ blocks, or (iii) $100 \times 100$ (all) pixels, as illustrated on the test image, (b,c), (d) or (e), respectively.
%%The pixels of a test image (a) are shuffled in $2 \times 4$ blocks (b,c), $10\times\10$ blocks, or $100\times\100$ (all) pixels (d). 
%Each of the three shufflings is then subject to topology estimation, and the resulting topology applied to reconstruct the test image (f,g,h).
%%The estimated sensor topology, applied to each of the three cases, in then used to reconstruct the test image (e,f,g).
%}
%\label{fig:results2}
%\end{figure}


\begin{figure}[t]
\centering
\begin{tabular}{ccc}
\includegraphics[height=3.0cm]{moon_view_6frames.eps} &
%\includegraphics[height=3.2cm]{topology_32_lines_unrot.eps} & 
\includegraphics[height=3.2cm]{topology_32_lines_unrot2.eps} & 
\includegraphics[height=2.7cm]{topology_32_lines_rot2.eps} \\
%\parbox{3cm}{\centering \small (a) Calibration frames 890, 2712, 4281 and 5263} &
\parbox{3cm}{\centering \small (a) 6 of 14784 calibration frames} &
\parbox{3cm}{\centering \small (b) Topology found} & 
\parbox{5cm}{\centering \small (c) Topology found, rotated (Eq.\ref{eq:rot_mirror}) and top-right zoomed} %\\ 
\vspace{3mm} \\
\end{tabular}
%
\begin{tabular}{ccccc}
\includegraphics[width=2.0cm]{car_case_35.eps} & 
\includegraphics[width=2.0cm]{face_shuffle2.eps} &
\includegraphics[width=2.0cm]{car_case_32.eps} & 
\includegraphics[width=2.0cm]{car_case_34.eps} & 
\includegraphics[width=2.0cm]{car_case_30.eps} \\
\parbox{2cm}{\centering \small (d) Test image} &
\parbox{2cm}{\centering \small (e) $2 \times 4$ permutation} &
\parbox{2cm}{\centering \small (f) $2 \times 4$ permutation} &
\parbox{2cm}{\centering \small (g) $10 \times 10$ permutation} &
\parbox{2cm}{\centering \small (h) $100 \times 100$ permutation} %\\
\vspace{3mm} \\
\end{tabular}
%
\begin{tabular}{ccc}
%\includegraphics[width=2.0cm]{topology_file_res02_case_32_v1.eps} & 
%\includegraphics[width=2.0cm]{topology_file_res02_case_34_v1.eps} & 
%\includegraphics[width=2.0cm]{topology_file_res02_case_30_v1.eps} \\
\includegraphics[width=2.0cm]{topology_find_res02_case_32_rot_mirror.eps} & 
\includegraphics[width=2.0cm]{topology_find_res02_case_34_rot_mirror.eps} & 
\includegraphics[width=2.0cm]{topology_find_res02_case_30_rot_mirror.eps} \\
\parbox{3.0cm}{\centering \small (i) Reconstruction after $2 \times 4$ permutation} &
\parbox{3.0cm}{\centering \small (j) Reconstruction after $10 \times 10$ permutation} &
\parbox{3.2cm}{\centering \small (k) Reconstruction after $100 \times 100$ permutation} %\\
\vspace{3mm} \\
\end{tabular}
\caption{
%Topology using minimum bounding box (note: 14784 frames).
%Topology estimation applied to the reconstruction of a shuffled $100 \times 100$ sensor.
%Topology estimation applied to reordering a shuffled sensor.
%Topology estimation applied to the reconstruction of an image.
Moon images used to estimate the topology of a $100 \times 100$ sensor (a).
Estimated topology after random permutation of the pixel-streams (b,c).
%Test image before shuffling (d).
%The pixels of the sensor are shuffled in (i) $2 \times 4$ blocks, (ii) $10 \times 10$ blocks, or (iii) $100 \times 100$ (all) pixels, as illustrated on the test image, (e,f), (g) or (h), respectively.
Test image before permutation (d).
%Permutations of pixel-streams in $2 \times 4$ blocks (e), $10 \times 10$ blocks, or $100 \times 100$ (all) pixels, illustrated on the test image, (f), (g) or (h), respectively.
%Each of the three permutations is then subject to topology estimation, and the resulting topology applied to reconstruct the test image (i,j,k).
Permutations of pixel-streams in $2 \times 4$ blocks, $10 \times 10$ blocks, or $100 \times 100$ (all) pixels, illustrated on the test image, (e,f,g,h).
Estimated topology applied to reconstruct the test image after the three permutations (i,j,k).
}
\label{fig:results2}
\end{figure}


In the first experiment the camera was pointed to the moon, in a full-moon night, to obtain calibration data (see Fig.~\ref{fig:results2}(a)). The data acquisition was performed at 24fps for about ten minutes (14784 frames), while panning and tilting the camera.
%
Figures~\ref{fig:results2}(b,c) show the estimated topology, approximately forming a regular square grid, close to the ground truth.
%
%Manifold learning toolbox modified to receive distances instead of the pixel streams (so we can use a correlation based distance instead of a simple euclidean distance)
% ^^ time delay inter-pixel is the most informative data?
%
A test image, acquired in daylight (Fig.~\ref{fig:results2}(d)), was then used to illustrate more clearly that the sequencing of the pixel-streams (see pixel permutations in Figs.~\ref{fig:results2}(e,f,g,h)) does not influence the perceptual quality of the estimated topology and image reconstruction (Figs.~\ref{fig:results2}(i,j,k)).
%
Despite having obtained the results in Figs.~\ref{fig:results2}(i,j,k) with different calibrations, and thus subject to different random selections of landmark pixels, the differences of the estimated topologies are small.
Using a 2D Procrustes, to register the three reconstructed topologies with a square 1-pixel-steps grid, resulted in 
%average relative pixel localization errors
%~\footnote{$e= \sum\nolimits_i {\sqrt {xi' - xi} /N/W}$, where $W$ is the image width} of $3.52\%$, $3.35\%$ and $3.92\%$, respectively. 
%of $11.14\%$, $11.08\%$ and $11.00\%$, and
inter-pixel (four nearest neighbors) distance-error distribution with a standard deviation of $0.566$, $0.563$ and $0.563$ pixels, respectively. %, around the optimal location.

%--- Exp2: calibration given a sequence of the garden
% two critical aspects: (i) uniform edge directions or uniform motion (vs longer sequences?), (ii) binarization to make sharper (lesser ambiguous) pixel-stream transitions
%.... uniform edge directions distribution similar to what was found in the moon sequence, hence the selection of a garden sequence
%From the bibliography it is known that 
%
%results: (i) gray level imply scaling of diagonal directions
%(ii) lost aspect ratio, maybe due to bias short sequence, not perfectly uniform distribution of pan, tilt and roll

\begin{figure}
\centering
%
\begin{tabular}{cccc}
\includegraphics[width=2.0cm]{garden_0123_crop_000.eps} & 
\includegraphics[width=2.0cm]{garden_0123_crop_025.eps} & 
\includegraphics[width=2.0cm]{garden_0123_crop_6105.eps} & 
\includegraphics[width=2.0cm]{garden_0123_crop_13209.eps} \\
\parbox{2cm}{\centering \small (a) Calibration frame 1} &
\parbox{2cm}{\centering \small (b) Calibration frame 25} &
\parbox{2cm}{\centering \small (c) Calibration frame 6105} &
\parbox{2cm}{\centering \small (d) Calibration frame 13209} \\
\end{tabular}
%
\begin{tabular}{cccc}
\vspace{3mm} \\
\includegraphics[width=2cm]{garden_0123_crop_6105_bw.eps} &
\includegraphics[width=2cm]{garden_0123_crop_13209_bw.eps} & \includegraphics[width=2cm]{topology_find_res02_case_31_150_lines.eps} & \includegraphics[width=2cm]{topology_find_res02_case_31_150_rot_mirror.eps} \\
%
\parbox{2cm}{\centering \small (e) Frame 6105 binarized} &
\parbox{2cm}{\centering \small (f) Frame 13209 binarized} &
\parbox{2cm}{\centering \small (g) Topology found} &
\parbox{2cm}{\centering \small (h) Frame 13209 reconstructed} \\
%
\vspace{3mm} \\
\end{tabular}
%
\caption{
Topology estimation applied to 17448 images random images of the University campus
}
\label{fig:results3}
\end{figure}


In the second experiment, the main purpose is to explore more general calibration scenarios, while keeping the topology estimation accurate.
%
In the moon sequence, edges are clearly defined and have directions distributed uniformly despite of the discretization due to the non-infinitesimal pixel size.
%
Hence we considered the garden %and car park 
scenario, Fig.~\ref{fig:results3}(a,b,c,d), where the vegetation also provides many edge directions. For this data set we filmed about twelve minutes, at 24fps, having acquired 17448 frames. In this case we do pan and tilt motions, as well as roll and translation. % although most of the edges are vertical and horizontal. %, 
%which can be further extended by rolling the camera, 
%and image binarization augments edges sharpness.
%
%After the binarization the 
Figure~\ref{fig:results3}(g) shows the topology reconstruction considering binary level pixel-streams. Some images used by the algorithm can be seen in Figs.~\ref{fig:results3}(e,f).
%
Despite the complexity of scene texture
%features in the environment observed 
we do not use a look up table as required in \cite{Grossmann10}. %One observes that the image corners are stretched out, which is due to an overestimation of inter-pixel distances in the vertical direction. This can be due to a bias in the dataset which may contain more horizontal edges than vertical.
%
%Fig g is rotated and mirrored
%
%Fig h is corrected 
%
%The rotation and mirror in 
%
%Observing Figs.~\ref{fig:results3}(g,h) simultaneously, one can detect that the topology suffered a mirror effect, this is explained because 
%
Note that Fig.~\ref{fig:results3}(g) is the direct result of the Landmark Isomap, while 
Fig.~\ref{fig:results3}(h) is the result after using Eq.\ref{eq:rot_mirror}, and thus show a detected and corrected mirror effect.
%
%which detects if the output of the multi scaling algorithm is affected by a mirror effect.
%
%By binarizing the images, Figs.~\ref{fig:results3}(f,g), one augments image edges distinctiveness and improves the accuracy of reconstruction, Fig.~\ref{fig:results3}(h). This beneficial effect of image binarization is in accordance with related work \cite{Grossmann10}, where was shown experimentally that reducing the brightness levels of images would allow using shorter calibration sequences.


------

Proposed a simple auto-calibration methodology for an arbitrary central projection camera, in this work an optic fibers bundle. Proved that the relationship between correlation and inter-pixel angle is approximately linear for a camera rotating in the center of a dark spherical-surface with a light circular (hub-cap) source

-------

\begin{figure}
\centering
\begin{tabular}{ccc}
\includegraphics[width=3cm]{face_case_35.eps} & 
\multicolumn{2}{c}{ \includegraphics[width=5cm]{face_shuffle.eps} }\\
(a) Test image & \multicolumn{2}{c}{(b) $4 \times 2$ shuffling } %\\
\vspace{3mm} \\
\includegraphics[width=3cm]{face_case_32.eps} & 
\includegraphics[width=3cm]{face_case_34.eps} & 
\includegraphics[width=3cm]{face_case_30.eps} \\
(c) $4 \times 2$ shuffling &
(d) $10 \times 10$ shuffling &
(e) $100 \times 100$ shuffling %\\
\vspace{3mm} \\
\includegraphics[width=3cm]{topology_file_res02_case_32.EPS} & \includegraphics[width=3cm]{topology_file_res02_case_34.EPS} & \includegraphics[width=3cm]{topology_file_res02_case_30.EPS} \\
(f) Reconstruction after & (g) Reconstruction after & (h) Reconstruction after \\
 $4 \times 2$ shuffling &
 $10 \times 10$ shuffling &
 $100 \times 100$ shuffling %\\
\vspace{3mm} \\
\end{tabular}
\caption{
Topology estimation applied to the reconstruction of a shuffled $100 \times 100$ sensor.
%Topology estimation applied to reordering a shuffled sensor.
%Topology estimation applied to the reconstruction of an image.
Test image before shuffling (a).
The pixels of the sensor are shuffled in (i) $4 \times 2$ blocks, (ii) $10 \times 10$ blocks, or (iii) $100 \times 100$ (all) pixels, as illustrated on the test image, (b,c), (d) or (e), respectively.
%The pixels of a test image (a) are shuffled in $4\times2$ blocks (b,c), $10\times\10$ blocks, or $100\times\100$ (all) pixels (d). 
Each of the three shufflings is then subject to topology estimation, and the resulting topology applied to reconstruct the test image (f,g,h).
%The estimated sensor topology, applied to each of the three cases, in then used to reconstruct the test image (e,f,g).
}
\label{fig:results2}
\end{figure}

\begin{figure}
\centering
\begin{tabular}{c}
\includegraphics[width=5.0cm]{panoramic_mosaic.eps}  
\end{tabular}
\caption{Scanned texture (a). Images, 3 out of 258, acquired with the fiber bundle (b). Topology corrected images (c). Mosaic built from all the258 images given the camera motion (d).}
\end{figure}



\section{Marginal PDF characterizing Pixel Densities}

Suppose we have two kinds of pixels densities characterizing the cable optic fibers (i) uniform in $uv$ (pixel coordinates), or constant marginal density for each circle, i.e. uniform in r. The uniform density in $uv$ implies a non uniform density in $r$, and vice versa (see figure\ref{fig:uniform_disp}).

\begin{figure}
\centering
\begin{tabular}{cc}
\includegraphics[width=6.0cm]{uniform_uv.eps} &
\includegraphics[width=6.0cm]{uniform_R.eps} \\
(a) & (b) \\
\includegraphics[height=4.0cm]{uniform_uv_hist.eps} &
\includegraphics[height=4.0cm]{uniform_R_hist.eps} \\
(c) & (d) 
\end{tabular}
\caption{
Uniform density in uv, i.e. $f_{uv}(x,y)=$const (a).
Uniform density considering just r, i.e. $f_r(r)=$const the marginal function obtained by accumulating for all values of $\alpha$ (b).
Histogram of (a) and (b) along the radius, considering a thin crowd for each $r$, shown in (c) and (d), respectively.
}
\label{fig:uniform_disp}
\end{figure}

%Let us start with the uniform density in uv. Noting that r=sqrt(u2+v2), the distribution corresponding to the uniform density in uv has a form of
%\begin{equation}
%f(r)= a*r*(u(r)-u(r-\max(r)))
%\end{equation}
%where $u(.)$ denotes the unit step / Heaviside function.


The probability density function of a random variable X is given as $f_X(x)$. It is possible to calculate the probability density function of some variable $Y= g(X)$. This is also called a change of variable and is in practice used to generate a random variable of arbitrary shape $f_{g(X)} = f_Y$ using a known (for instance uniform) random number generator.

\begin{equation}
f_Y(y)=\left | \frac{d}{dy} (g^{-1}(y)) \right | f_x(g^{-1}(y))
\label{eq:wiki_copy}
\end{equation}

%\begin{tabular} {cc}
%In case of uniform distribution in R we have & The theoretical solution for the histograms with a uniform distribution in uv is, where \\
%1 & 2
%\end{tabular}

Looking at the case we assume uniform density in uv, we have to convert the density in to the polar coordinates, the transformation from Cartesian coordinates into polar is:

\begin{equation} 
\begin{array}{c}
x=R\cos(\alpha) \\
y=R\sin(\alpha)
\end{array}
\label{eq:xy2polar}
\end{equation}
%
considering $J$ the Jacobian matrix of equation~\ref{eq:xy2polar} we have:

\begin{equation}
|J(r,\alpha)|= \left |
\begin{array} {cc}
\cos(\alpha) & \sin(\alpha) \\
-r\sin(\alpha) & r\cos(\alpha)
\end{array} \right |=r
\end{equation}

As we assume that we have a uniform density in uv, we can consider that it is one, $f_{ux}(x,y)=1$, replacing the the density and the Jacobian in equation \ref{eq:wiki_copy} the result is:

\begin{equation}
f_{r\alpha}(r,\alpha)=|J(r,\alpha)|f_{ux}(r\cos(\alpha), r\sin(\alpha))=r
\end{equation} 


%The distribution made by a density function is calculated using an integral in the area wanted, however since we are using the polar coordinates we have to take into account the transformation $r$.
%%
%
%\begin{equation}
 %F(r, \alpha) = \int^{2\pi}_{0} \int^{R}_0 f(r, \alpha) r\ dr d\alpha
%\label{eq:int_circle}
%\end{equation}
%%
%$f$ corresponds to the density function, in the case of uniform distribution in $uv$ the $f^{uv}(r,\alpha)=c$, being $c$ the value of the uniform distribution and the superscript represents the distribution, replacing in eq~\ref{eq:int_circle} the distribution will be $c\pi R^2$.
%%
%For the uniform distribution in $R$ the density function corresponds to $f^R(r, \alpha)=\frac{a}{2\pi r}$, being $a$ the number of points per radius $r$. Replacing in  eq~\ref{eq:int_circle} the distribution will be $a R$.
%However since we want only the marginal distribution in $r$, $f_R(r)=\frac{F(r,\alpha)}{d_r}$. In the case of the uniform distribution in $uv$ the marginal distribution will be 
%%
%\begin{equation}
%f^{uv}_R(r)=c2\pi r
%\end{equation}
%%
%in the case of uniform distribution in R it will be a constant 
%%
%\begin{equation}
%f^{R}_R(r)=a.
%\end{equation}

Since the definition of marginalization is:
%
\begin{equation}
f_X(x)=\int_{-\inf}^{\inf} f(x,y)\ dy
\end{equation}
%
one can compute the marginal density in $R$.
%
\begin{equation}
f_R(r)=\int^{2\pi}_0 f(r,\alpha) d\alpha =2\pi r
\end{equation}
%
Note that for the second case we already know $f_R(r)$ since we consider that is has an uniform marginal density in $R$.

Figures~\ref{fig:uniform_disp}(c) and (d) show the marginal density of in both cases $f^{\{uv,R\}}_R(r)$, when we have a uniform in uv or when it is in R.

%\begin{figure}
%\centering
%\begin{tabular}{cc}
%\includegraphics[width=6.0cm]{uniform_uv_hist.eps} &
%\includegraphics[width=6.0cm]{uniform_R_hist.eps} 
%\end{tabular}
%\caption{On the left is the distribution of R when we have a uniform distribution in uv, on the right is the distribution of R when there is a uniform distribution in R.}
%\label{fig:histogram}
%\end{figure}


\section{Lens Effect and Raxel Densities}

------

Grossberg and Nayar\cite{Raxels} defined the concept of raxels, raxels is a mathematical abstraction of the position of the light sensor. Instead of the real position of the sensor, raxels are assumed to be along the direction of the chief ray associated to the sensor, they can be catheterized as a $3D$ position and a direction vector, as it can be seen in Fig.\ref{fig:setup}(d).
 
-----

A raxels as stated before, is catheterized by a $3D$ position, $p$, and a vector of orientation, $q$. However since we are working with central cameras, one considers that all light rays converge to the same point, thus, making $p_i=p_j$ for any pair of raxels. Therefore, one can ignore the position of the raxels, since the only useful information is contained in the direction vector $q$. As a direction vector, $q$, one can assume that all the vectors have the same norm, this assumption removes one degree of freedom, which allow us to represent $q$ with only two angles, $(\Omega,\mu)$, as it can be seen in figure~\ref{fig:def_raxels}.
%
\begin{figure}
\centering
\begin{tabular}{c}
\includegraphics[width=6.0cm]{raxel_v1.eps}
\end{tabular}
\caption{Characterizing raxels.}
\label{fig:def_raxels}
\end{figure}
%
Let $T$ be the transformation produced by a lens type, the transformation from raxels to pixels can be represented as:
\begin{equation}
\left [ \begin{array}{c} u \\ v \end{array} \right ] = h^{-1} (\Omega)
\left [ \begin{array}{c} \cos (\mu) \\ \sin(\mu) \end{array} \right ]
\end{equation}

The marginal density would be the solution if we had no lens transformation, however the transformation created by the lens will change the marginal density. Once again we can use Eq.~\ref{eq:wiki_copy} to change the density function, however in this case we only interested in the marginal density. For the cases we show so, far 3 lens and 2 possible distributions, we have 6 solution, the results can be seen in the next table:

\begin{table*}[h] % place table ��here�� or at bottom of page
\centering
\begin{tabular}{|p{1in}|p{1.5in}|p{2.5in}|}
\hline
  & In case of uniform distribution in R we have:  &  The theoretical solution for the histograms with a uniform distribution in uv is, where:  \\
\hline
 & $f_x=f^{R}_R(r)\propto 1$ &  $f_x=f^{uv}_R(r)\propto r$\\
\hline
$g(\Omega)^{-1}=\tan(\Omega)$ & $f_y=\sec(\Omega)^2$ & $f_y=\sec(\Omega)^2\tan(\Omega)$ \\
$g(\Omega)^{-1}=\Omega$ & $f_y=1$ & $f_y=\Omega$ \\
$g(\Omega)^{-1}=\sin(\Omega)$ & $f_y=\cos(\Omega)$ & $f_y=\cos(\Omega)\sin(\Omega)$ \\
\hline
\end{tabular}
\end{table*}

In figure~\ref{fig:histogram_r} shows simulated and theoretical results, when the sensor has a uniform marginal distribution along R.
%
\begin{figure}
\centering
\begin{tabular}{cc}
\includegraphics[width=6.0cm]{uniform_R_hist_perspective.eps} &
\includegraphics[width=6.0cm]{uniform_R_hist_ortogonal.eps} 
\end{tabular}
\caption{Histogram of orthogonal projection (right) and from perspective (left) at red are the theoretical values while in blue are the synthetic values. }
\label{fig:histogram_r}
\end{figure}
%
Note the equidistant projection is not shown in here but the experimental vs theoretical results are also close.

In the next cases it is show a square, in the topology, instead of a circle, because it is easier to spot the differences between each case, however the histogram was made only using the biggest circle that could fit in the topology.
For the case we have a uniform distribution in uv ($f_X=f^{uv}_R(r)$):


\begin{figure}
\centering
\begin{tabular}{ccc}
\includegraphics[width=5.0cm]{uniform_uv_top_perspective.eps} & 
\includegraphics[width=5.0cm]{uniform_uv_top_equidistant.eps} &
\includegraphics[width=5.0cm]{uniform_uv_top_ortogonal.eps} \\
\includegraphics[width=5.0cm]{uniform_uv_hist_perspective.eps} &
\includegraphics[width=5.0cm]{uniform_uv_hist_equidistant.eps} &
\includegraphics[width=5.0cm]{uniform_uv_hist_ortogonal.eps} 
\end{tabular}
\caption{At top, is the topology created by lens distortion when we have a uniform sensor, at the bottom are the histogram corresponding to each lens, in red are the theoretical histograms and in blue the synthetic data. From left to right, perspective, equidistant and orthogonal lens.}
%\label{fig:histogram}
\end{figure}


Algorithm observing the radial distribution of the topology until its maximum value, since a topology may not be circular. 

$h(\Omega) = f(\Omega)$ with $\Omega \in [0\ f^{-1}(\max (f(.))]$

A quadratic curve is estimated from $h$ using MSR, and depending on the value of the second order term we can estimate the lens our camera has. Choosing the right lens consists in looking to the second order term. The first case that we look at, is the equidistant lens, if the second order term has a value lower, in modulus, than $10\%$ of the 1st order value the curve is considered to belong to a equidistant camera.  If the second order term is negative the lens on the camera is a orthogonal lens, otherwise it is a perspective lens.


\section{Results}




\begin{figure}
\centering
\begin{tabular}{cc}

\includegraphics[width=.3\columnwidth]{equi_orig_img.eps} & 

\includegraphics[width=.45\columnwidth]{equidistance_top.eps}  \\
\includegraphics[width=.3\columnwidth]{equi_top_img.eps} &
\includegraphics[width=.3\columnwidth]{equidistance_distribution.eps} 

\end{tabular}
\caption{Equidistance camera}
\label{fig:def_raxels}
\end{figure}


\section{Conclusions and Future work}



%\section*{Acknowledgments}
%
%%\noindent To be filled.
%\noindent This work has been partially supported
%by the FCT project PEst-OE / EEI / LA0009 / 2011,
%by the FCT project PTDC / EEACRO / 105413 / 2008  DCCAL,
%and
%by the project High Definition Analytics (HDA), QREN - I\&D em Co-Promo\c{c}\~{a}o 13750.
%%
%We thank Jo\~{a}o Mendanha Dias, IST Lisbon, the help to build the microscopic input, and Ricardo Nunes, IST/ISR Lisbon, the help to build the setup based on a cable of optic fibers.

\bibliographystyle{splncs03}
\bibliography{secz}
\end{document}
