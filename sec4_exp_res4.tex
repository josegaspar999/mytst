\begin{figure}
\centering
\begin{tabular}{ccc}
\includegraphics[width=3cm]{face_case_35.eps} & 
\multicolumn{2}{c}{ \includegraphics[width=5cm]{face_shuffle.eps} }\\
(a) Test image & \multicolumn{2}{c}{(b) $4 \times 2$ shuffling } %\\
\vspace{3mm} \\
\includegraphics[width=3cm]{face_case_32.eps} & 
\includegraphics[width=3cm]{face_case_34.eps} & 
\includegraphics[width=3cm]{face_case_30.eps} \\
(c) $4 \times 2$ shuffling &
(d) $10 \times 10$ shuffling &
(e) $100 \times 100$ shuffling %\\
\vspace{3mm} \\
\includegraphics[width=3cm]{topology_file_res02_case_32.EPS} & \includegraphics[width=3cm]{topology_file_res02_case_34.EPS} & \includegraphics[width=3cm]{topology_file_res02_case_30.EPS} \\
(f) Reconstruction after & (g) Reconstruction after & (h) Reconstruction after \\
 $4 \times 2$ shuffling &
 $10 \times 10$ shuffling &
 $100 \times 100$ shuffling %\\
\vspace{3mm} \\
\end{tabular}
\caption{
Topology estimation applied to the reconstruction of a shuffled $100 \times 100$ sensor.
%Topology estimation applied to reordering a shuffled sensor.
%Topology estimation applied to the reconstruction of an image.
Test image before shuffling (a).
The pixels of the sensor are shuffled in (i) $4 \times 2$ blocks, (ii) $10 \times 10$ blocks, or (iii) $100 \times 100$ (all) pixels, as illustrated on the test image, (b,c), (d) or (e), respectively.
%The pixels of a test image (a) are shuffled in $4\times2$ blocks (b,c), $10\times\10$ blocks, or $100\times\100$ (all) pixels (d). 
Each of the three shufflings is then subject to topology estimation, and the resulting topology applied to reconstruct the test image (f,g,h).
%The estimated sensor topology, applied to each of the three cases, in then used to reconstruct the test image (e,f,g).
}
\label{fig:results2}
\end{figure}
